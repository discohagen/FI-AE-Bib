\subsection{Eigene Rolle im Betrieb}

Die eigene Rolle im Betrieb ist vorwiegend durch Rechte und Pflichten in der Ausbildung geprägt. Vor allem herrschen hier die Grundsätze des Individualrechts:

\subsubsection{Berufsbildungsgesetz (BBiG)}

Die Berufsbildung wird in Betrieben und in Berufsschulen kooperativ durchgeführt (§2 Abs. 1 und 2 BBiG).

Die Form und geforderte Inhalte einer Ausbildungsordnung ist definiert (§5 BBiG). Die Ausbildungsordnung für Fachinformatiker ist weiter unten unter FIAusbV beschrieben.

Es muss ein Ausbildungsvertrag geschlossen werden (§10 BBiG). Dieser muss folgendes beinhalten (§11 BBiG):

\begin{itemize}
    \item Name und Anschrift der Vertragsparteien
    \item Art, sachliche und zeitliche Gliederung und Ziel der Ausbildung
    \item Beginn und Dauer
    \item Ausbildungsstätte und Ausbildungsnahmen außerhalb
    \item tägliche Arbeitszeit
    \item Probezeit
    \item Vergütung
    \item Umgang mit Überstunden
    \item Urlaub
    \item Voraussetzungen für Kündigung
    \item Form des Ausbildungsnachweises
\end{itemize}

Folgende Vereinbarung sind in einer Ausbildung nichtig (§12 BBiG):

\begin{itemize}
    \item Verpflichtung zur Übernahme (außer 6 Monate vor Ausbildungsende)
    \item Verpflichtung zur Entschädigungszahlung für die Ausbildung
    \item Vertragsstrafen
    \item Ausschluss oder Beschränkung von Schadensersatzansprüchen inkl. der Festsetzung von Pauschalen bei Schadensersatz
\end{itemize}

Pflichten des Auszubildenden sind u.a. (§13 BBiG):

\begin{itemize}
    \item sorgfältig Aufgaben auszuführen
    \item Ausbildungsmaßnahmen wahrzunehmen, für welche Sie freigestellt werden
    \item Weisungen befolgen
    \item Ausbildungsstättenordnung beachten
    \item Werkzeug, Maschinen und sonstiges pfleglich behandeln
    \item Schweigepflicht über Betriebsgeheimnisse
    \item schriftlichen oder elektronischen Ausbildungsnachweis führen
\end{itemize}

Pflichten des Ausbildenden sind u.a. (§14 BBiG):

\begin{itemize}
    \item nach bestem Gewissen für den Beruf auszubilden
    \item selbst auszubilden oder ausdrücklich einen Ausbilder beauftragen
    \item Ausbildungsmittel kostenlos zur Verfügung stellen
    \item Auszubildende zum Besuch der Berufsschule anzuhalten
    \item Charakter des Auszubildenden fördern und körperliche Gefahren vermeiden
    \item Auszubildende zum Führen des Ausbildungsnachweises anzuhalten und diesen regelmäßig durchzusehen
    \item nur Aufgaben stellen, welche dem Ausbildungszweck dienen und den körperlichen Kräften des Auszubildenden angemessen sind
\end{itemize}

Auszubildende sind für die Berufsschule und Prüfungen freizustellen (§15 Abs. 1 und 2 BBiG). Für volljährige Auszubildende gilt:

\begin{table}[H]
    \centering
    \begin{tabularx}{\textwidth}{|>{\centering\arraybackslash}X|>{\centering\arraybackslash}X|}
        \hline
        Situation und Freistellung                                                                   & Anrechnung der Arbeitszeit                                                                          \\
        \hline
        Berufsschulunterricht                                                                        & Unterrichts- und Pausenzeit und notwendige Wegzeiten zwischen Berufsschule und Ausbildungsstätte    \\
        \hline
        ein Berufsschultag in der Woche mit mehr als 5 Unterrichtsstunden á 45 Min                   & durchschnittliche tägliche Arbeitszeit                                                              \\
        \hline
        Berufsschulwochen mit einem planmäßigen Blockunterricht von mindestens 25 Stunden an 5 Tagen & durchschnittliche wöchentliche Arbeitszeit                                                          \\
        \hline
        Prüfungen und Ausbildungsmaßnahmen                                                           & Zeit der Teilnahme inkl. Pausen und notwendig Wegzeiten zwischen Teilnahmeort und Ausbildungsstätte \\
        \hline
        Arbeitstag vor der AP 2                                                                      & durchschnittliche tägliche Arbeitszeit                                                              \\
        \hline
    \end{tabularx}
    \caption{Freistellung, Anrechnung}
    \label{tab:freistellung}
\end{table}

Der Ausbildende hat bei Beendigung ein Arbeitszeugnis auszustellen (§16 BBiG).

Auszubildende haben ein Anrecht auf eine Mindestvergütung mit jedem Lehrjahr steigend (§17 BBiG).

Die Probezeit darf zwischen einem und vier Monaten dauern (§20 BBiG).

Die Ausbildung endet mit Ablauf der Ausbildungsdauer oder bei bestehen der Abschlussprüfung mit Bekanntgabe der Ergebnisse (§21 Abs. 1 und 2 BBiG). Der Auszubildende kann bei nicht bestehen Verlangen das Ausbildungsverhältnis bis zur nächstmöglichen Prüfungswiederholung zu verlängern, maximal aber ein Jahr (§21 Abs. 3 BBiG).

Während der Probezeit kann jederzeit und ohne Frist gekündigt werden (§22 Abs. 1 BBiG). Nach der Probezeit darf nur aus einem wichtigen Grund und ohne Frist gekündigt werden oder vom Auszubildenden mit einer Frist von vier Wochen, wenn Sie die Ausbildung aufgeben oder eine andere Berufstätigkeit ausüben wollen (§22 Abs. 2 BBiG). Kündigungen müssen schriftlich sein und außerhalb der Probezeit den Kündigungsgrund beinhalten (§22 Abs. 3 BBiG). Eine Kündigung aus wichtigem Grund ist unwirksam, wenn dieser dem Kündigungsberechtigten länger als zwei Woche bekannt ist, außer es ist ein Güteverfahren eingeleitet, welches die Frist hemmt (22 Abs. 4 BBiG).

Werden Auszubildende nach Abschluss der Ausbildung beschäftigt ohne ausdrückliche Vereinbarung, so liegt automatisch ein Arbeitsverhältnis auf unbestimmte Zeit vor (§24 BBiG).

Die Abschlussprüfung kann bis zu zweimal wiederholt werden und ist für Auszubildende gebührenfrei. Es muss ein Zeugnis ausgestellt werden (§37 BBiG).

\textbf{Fachinformatikerausbildungsverordnung (FIAusbV)}

Die Ausbildung dauert 3 Jahre (§2 FIAusbV).

Gliederung in die Fachrichtungen Anwendungsentwicklung, Systemintegration, Daten- und Prozessanalyse und Digitale Vernetzung (§4 Abs. 1 Satz 2 FIAusbV).

Regelungen zur Abschlussprüfung finden sich in den §§ 7 bis 41 FIAusbV.

\textbf{BUrlG}

U.a. als volljähriger Auszubildender hat man einen Anspruch auf bezahlten Urlaub von mindestens 24 Werktagen pro vollem Jahr (§§ 1 bis 3 BUrlG).

\textbf{ArbSchG}

Der Arbeitgeber hat Gefahren für den Arbeitnehmer bestmöglich zu vermeiden oder gering zu halten, in dem er Maßnahmen des Arbeitsschutzes trifft und generell eine Verbesserung von Sicherheit und Gesundheitsschutz anstrebt. Der Arbeitgeber hat den Arbeitnehmer zu unterweisen und der Arbeitnehmer hat sich möglichst an die Unterweisungen und Weisungen für seinen Schutz zu halten.

\textbf{ArbZG}

Die werktägliche Arbeitszeit ist max. acht Stunden. Sie kann auf zehn Stunden verlängert werden, wenn die durchschnittliche Arbeitszeit innerhalb von 24 Wochen acht Stunden werktäglich nicht überschreitet (§3 ArbZG).

Ab einer Arbeitszeit von sechs Stunden bis zu neun Stunden sind voraus feststehende Ruhepausen von insgesamt mindestens 30 Min. und ab einer Arbeitszeit ab neun Stunden Ruhepausen von insgesamt mindestens 45 Min. einzulegen. Eine Ruhepause muss min. 15 Minuten betragen. Es darf nicht länger als sechs Stunden ohne Ruhepause gearbeitet werden (§4 ArbZG).

Zwischen den Arbeitszeiten muss eine Ruhezeit von mindestens elf Stunden liegen (§5 Abs. 1 ArbZG).

Es gilt ein generelles Beschäftigungsverbot an Sonn- und Feiertagen (§9 Abs. 1 ArbZG).

Es existieren definierte Ausnahmen und abweichende Regelungen.

\textbf{MuSchG}

\textbf{JArbSchG}

\textbf{BEEG}

\textbf{SGB IX}

\textbf{AGG}

\textbf{Sonstige}

Es sind u.a. auch das Kündigungsschutzgesetz (KSchG) und Arbeitsstättenverordnung (ArbStättV) zu beachten.