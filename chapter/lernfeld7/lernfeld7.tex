\chapter{Lernfeld 7: Cyber-physische Systeme ergänzen}

\textbf{Die Schülerinnen und Schüler verfügen über die Kompetenz, die physische Welt und
IT-Systeme funktional zu einem cyber-physischen System zusammenzuführen.}

Die Schülerinnen und Schüler \textbf{analysieren} ein cyber-physisches System bezüglich eines
Kundenauftrags zur Ergänzung und Inbetriebnahme weiterer Komponenten.

Sie \textbf{informieren} sich über den Datenfluss an der Schnittstelle zwischen physischer Welt
und IT-System sowie über die Kommunikation in einem bestehenden Netzwerk. Sie verschaffen sich einen Überblick über die Energie-, Stoff- und Informationsflüsse aller am System beteiligten Geräte und Betriebsmittel.

Die Schülerinnen und Schüler \textbf{planen} die Umsetzung des Kundenwunsches, indem sie Kriterien für die Auswahl von Energieversorgung, Hardware und Software (Bibliotheken, Protokolle) aufstellen. Dazu nutzen sie Unterlagen der technischen Kommunikation und passen
diese an.

Sie \textbf{führen} Komponenten mit dem cyber-physischen System funktional \textbf{zusammen}.

Sie \textbf{prüfen} systematisch die Funktion, messen physikalische Betriebswerte, validieren den
Energiebedarf und protokollieren die Ergebnisse.

Die Schülerinnen und Schüler \textbf{reflektieren} den Arbeitsprozess hinsichtlich möglicher Optimierungen und diskutieren das Ergebnis in Bezug auf Betriebssicherheit und Datensicherheit.