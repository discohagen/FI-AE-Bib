\chapter{Lernfeld 6: Serviceanfragen bearbeiten}

\textbf{Die Schülerinnen und Schüler verfügen über die Kompetenz, Serviceanfragen einzuordnen, Fehlerursachen zu ermitteln und zu beheben.}

Die Schülerinnen und Schüler nehmen Serviceanfragen entgegen (direkter und indirekter
Kundenkontakt). Sie \textbf{analysieren} Serviceanfragen und prüfen deren vertragliche Grundlage
(Service-Level-Agreement). Sie ermitteln die Reaktionszeit und dokumentieren den Status
der Anfragen im zugrundeliegenden Service-Management-System.

Durch systematisches Fragen \textbf{ordnen} die Schülerinnen und Schüler Serviceanfragen unter
Berücksichtigung des Support-Levels und fachlicher Standards \textbf{ein}.

Sie \textbf{ermitteln} Lösungsmöglichkeiten im Rahmen des Support-Levels. Auf dieser Basis \textbf{bearbeiten} sie das Problem und dokumentieren den Bearbeitungsstatus. Sie kommunizieren
mit den Prozessbeteiligten situationsgerecht, auch in einer Fremdsprache, und passen sich
den unterschiedlichen Kommunikationsanforderungen an (Kommunikationsmodelle, Deeskalationsstrategien).

Sie \textbf{reflektieren} den Bearbeitungsprozess der Serviceanfragen und ihr Verhalten in Gesprächssituationen. Die Schülerinnen und Schüler diskutieren die Servicefälle und schlagen
Maßnahmen zur Qualitätssteigerung vor.

\section{IT Service}

\subsection{ITSM - IT Service Management}

\subsection{ITIL - IT Infrastructure Library}

\subsection{Ticketsysteme}

\subsection{SLA - Service Level Agreements}

\subsection{Eisenhauer Matrix}

\subsection{Prozesskostenkalkulation}

\section{Projektplanung}

\subsection{Projektmanagement}

\subsection{4-Phasen Modell}

\subsection{Problemanalyse}

\subsection{Projektcanvas}

\subsection{Zielformulierung (SMART)}

\subsection{Risikoanalyse}

\subsection{Projektstrukturplan und Arbeitspakete}

\subsection{Gantt-Diagramm}

\subsection{Netzplantechnik}

Der Netzplan setzt eine Vorgangsliste (IDs, Vorgang, Dauer, Vorgänger) voraus. Vorgangsknoten werden im Netzplan durch Pfeile in Abhängigkeit dargestellt. Der kritische Pfad ergibt sich aus der geringsten Projektzeit, erkennbar an den Vorgängen ohne Puffer.

\begin{figure}
    [H]
    \centering
    \begin{tabular}{|c|c|c|}
        \multicolumn{1}{c}{FAZ} & \multicolumn{1}{c}{}         & \multicolumn{1}{c}{FEZ} \\\hline
        ID                      & \multicolumn{2}{c|}{Vorgang}                           \\\hline
        D                       & GP                           & FP                      \\\hline
        \multicolumn{1}{c}{SAZ} & \multicolumn{1}{c}{}         & \multicolumn{1}{c}{SEZ} \\
    \end{tabular}
    \caption{Vorgangsknoten}
\end{figure}

\begin{itemize}
    \item D = Dauer
    \item FAZ = Frühester Anfangszeitpunkt (Bei erstem Vorgang null, sonst größter FEZ der Vorgänger)
    \item FEZ = Frühester Endzeitpunkt (FAZ + D)
    \item SAZ = Spätester Anfangszeitpunkt (SEZ - D)
    \item SEZ = Spätester Endzeitpunkt (Bei letztem Vorgang null, sonst kleinster SAZ der Nachfolger)
    \item GP = Gesamtpuffer (Summe der Puffer der Nachfolger; SAZ - FAZ)
    \item FP = Freier Puffer (Puffer des Vorgangs; kleinster FAZ der Nachfolger - FEZ)
\end{itemize}

\begin{table}
    [H]
    \centering
    \begin{tabular}{|c|l|c|c|}
        \hline
        \textbf{ID} & \textbf{Vorgang}                & \textbf{Dauer} & \textbf{Vorgänger} \\\hline
        1           & Infrastruktur ermitteln         & 1              & -                  \\\hline
        2           & Arbeitsplatzbedarf ermitteln    & 2              & 1                  \\\hline
        3           & Netzwerkplan entwerfen          & 1              & 1                  \\\hline
        4           & Peripheriebedarf ermitteln      & 1              & 3                  \\\hline
        5           & Hardware PC + Server beschaffen & 4              & 2                  \\\hline
        6           & Software beschaffen             & 2              & 5                  \\\hline
        7           & Netzwerkzubehör beschaffen      & 2              & 3; 5               \\\hline
        8           & Peripherie beschaffen           & 1              & 4                  \\\hline
        9           & Hardware PC + Server aufbauen   & 6              & 5                  \\\hline
        10          & Server installieren             & 3              & 9                  \\\hline
        11          & Netzwerk aufbauen               & 5              & 7                  \\\hline
        12          & PC-Image anlegen                & 1              & 6; 10; 11          \\\hline
        13          & Peripherie anschließen          & 1              & 8; 10; 11          \\\hline
        14          & Netzwerkplan dokumentieren      & 2              & 13                 \\\hline
        15          & Server-Image anlegen            & 1              & 10                 \\\hline
        16          & PC-Remote installieren          & 1              & 12                 \\\hline 17 & Gesamtdokumentation erstellen & 3 & 14; 15; 16 \\\hline
    \end{tabular}
    \caption{Beispiel Vorgangsliste}
\end{table}

\begin{figure}[H]
    \centering
    \includesvg[width=\textwidth,inkscapelatex=false]{figures/Netzplan.drawio.svg}
    \caption{Beispiel Netzplan anhand der Vorgangsliste}
\end{figure}
\FloatBarrier