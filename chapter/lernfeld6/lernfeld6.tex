\chapter{Lernfeld 6: Serviceanfragen bearbeiten}

\textbf{Die Schülerinnen und Schüler verfügen über die Kompetenz, Serviceanfragen einzuordnen, Fehlerursachen zu ermitteln und zu beheben.}

Die Schülerinnen und Schüler nehmen Serviceanfragen entgegen (direkter und indirekter
Kundenkontakt). Sie \textbf{analysieren} Serviceanfragen und prüfen deren vertragliche Grundlage
(Service-Level-Agreement). Sie ermitteln die Reaktionszeit und dokumentieren den Status
der Anfragen im zugrundeliegenden Service-Management-System.

Durch systematisches Fragen \textbf{ordnen} die Schülerinnen und Schüler Serviceanfragen unter
Berücksichtigung des Support-Levels und fachlicher Standards \textbf{ein}.

Sie \textbf{ermitteln} Lösungsmöglichkeiten im Rahmen des Support-Levels. Auf dieser Basis \textbf{bearbeiten} sie das Problem und dokumentieren den Bearbeitungsstatus. Sie kommunizieren
mit den Prozessbeteiligten situationsgerecht, auch in einer Fremdsprache, und passen sich
den unterschiedlichen Kommunikationsanforderungen an (Kommunikationsmodelle, Deeskalationsstrategien).

Sie \textbf{reflektieren} den Bearbeitungsprozess der Serviceanfragen und ihr Verhalten in Gesprächssituationen. Die Schülerinnen und Schüler diskutieren die Servicefälle und schlagen
Maßnahmen zur Qualitätssteigerung vor.

% Netzplantechnik