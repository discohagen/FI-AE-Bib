\section{IPv4}

IPv4 ist die vierte Version des Internet Protokolls und die erste, welche erfolgreich weltweit eingesetzt wurde und wird. Es gehört zur Internetprotokollfamilie erklärt durch TCP/IP und wird der TCP/IP Schicht für Internet bzw. der OSI-Schicht 3 (Netzwerk/ Vermittlung) zugeordnet. Es ist ein Protokoll zu Wegfindung in einem Netzwerk und bietet hierarchische Adressen.

IPv4 benutzt 32-Bit-Adressen. Eine Häufige Darstellungsart ist die Dezimalpunktschreibweise (decimal dotted), welche die Adresse in 4 Bytes teilt und diese durch Punkte getrennt dezimal darstellt. Diese Schreibweise ist menschlich besser lesbar aber sperrig bei Subnetting, welches innerhalb der sog. Oktette stattfindet. Der IPv4-Adressraum bietet in einem Gesamtnetz Platz für 4,3 Milliarden Adressen.

Die Internet Assigned Numbers Authority (IANA) vergibt IP-Adressen. Ursprünglich wurden die Adressen in Klassen eingeteilt. Dies ist aber wegen der Knappheit der IPv4-Adressen veraltet.

\begin{table}
    [H]
    \centering
    \begin{tabulary}{\textwidth}{|l|l|L|L|l|l|}
        \hline
        \textbf{Klasse} & \textbf{Präfix} & \textbf{Bereich}            & \textbf{Netzmaske}  & \textbf{Anzahl Netze} & \textbf{Anzahl Adressen} \\\hline
        A               & 0               & 0.0.0.0 - 127.255.255.255   & 255.0.0.0 (/8)      & 128                   & 16.777.216               \\\hline
        B               & 10              & 128.0.0.0 - 191.255.255.255 & 255.255.0.0 (/16)   & 16.384                & 65.536                   \\\hline
        C               & 110             & 192.0.0.0 - 223.255.255.255 & 255.255.255.0 (/24) & 2.097.152             & 256                      \\\hline
    \end{tabulary}
    \caption{Netzklassen}
\end{table}

Heutzutage gelten private Netze. IPv4-Adressen in diesen Bereichen werden nicht ins Internet geroutet und können daher in verschiedenen privaten Netzen gleichzeitig verwendet werden.

\begin{table}
    [H]
    \centering
    \begin{tabulary}{\textwidth}{|L|L|L|}
        \hline
        \multicolumn{2}{|c|}{\textbf{Bereich}} & \textbf{Anzahl Adressen} \\\hline
        10.0.0.0/8 & 10.0.0.0 - 10.255.255.255 & 16.777.216 \\\hline
        172.16.0.0/12 & 172.16.0.0 - 172.31.255.255 & 1.048.576 \\\hline
        192.168.0.0/16 & 192.168.0.0 - 192.168.255.255 & 65.536 \\\hline
    \end{tabulary}
    \caption{Private Netze}
\end{table}

\begin{table}
    [H]
    \centering
    \begin{tabular}{|l|l|}
        \hline
        \textbf{Adresse(n)} & \textbf{Bedeutung}         \\\hline
        0.0.0.0             & Platzhalter                \\\hline
        100.64.0.0/10       & Carrier-Grade NAT          \\\hline
        127.0.0.0/8         & Loopback                   \\\hline
        169.254.0.0/16      & Zeroconf (APIPA)           \\\hline
        192.0.2.0/24        & Dokumentation \& Beispiele \\\hline
        198.18.0.0/15       & Benchmarking               \\\hline
        198.51.100.0/24     & Dokumentation \& Beispiele \\\hline
        203.0.113.0/24      & Dokumentation \& Beispiele \\\hline
        224.0.0.0/4         & IPv4-Multicast             \\\hline
        255.255.255.255     & Broadcast                  \\\hline
    \end{tabular}
    \caption{Übersicht spezieller Adressen}
\end{table}

\subsection{Subnetting}

- Netzmaske / Subnetzmaske
- Netzwerkadresse / Netzadresse
- Broadcastadresse
- Max-Hosts
- Adressierungsarten: Unicast, Broadcast, Multicast, Anycast
- NAT