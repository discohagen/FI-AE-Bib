\section{IPv4}

IPv4 ist die vierte Version des Internet Protokolls und die erste, welche erfolgreich weltweit eingesetzt wurde und wird. Es gehört zur Internetprotokollfamilie erklärt durch TCP/IP und wird der TCP/IP Schicht für Internet bzw. der OSI-Schicht 3 (Netzwerk/ Vermittlung) zugeordnet. Es ist ein Protokoll zu Wegfingung in einem Netzwerk und bietet hierarchische Adressen.

IPv4 benutzt 32-Bit-Adressen. Eine Häufige Darstellungsart ist die Dezimalpunktschreibweise (decimal dotted), welche die Adresse in 4 Bytes teilt und diese durch Punkte getrennt dezimal darstellt. Diese Schreibweise ist menschlich besser lesbar aber sperrig bei Subnetting, welches innerhalb der sog. Oktette stattfindet. Der IPv4-Adressraum bietet in einem Gesamtnetz Platz für 4,3 Milliarden Adressen.

- Netzmaske / Subnetzmaske
- Netzwerkadresse / Netzadresse
- Broadcastadresse
- Max-Hosts
- Adressierungsarten: Unicast, Broadcast, Multicast, Anycast
- veraltete Netzklassen (A: 0.0.0.0 - 127.255.255.255, B: 128.0.0.0 - 191.255.255.255, C: 192.0.0.0 - 223.255.255.255)
- Private Bereiche (10.0.0.0/8, 172.16.0.0/12, 192.168.0.0/16)
- Spezielle Adressen (127.0.0.0/8 Loopback, 169.254.0.0/16 APIPA, 192.0.2.0 /24 Dokumentation und Beispiele)