\section{IPv4}

IPv4 ist die vierte Version des Internet Protokolls und die erste, welche erfolgreich weltweit eingesetzt wurde und wird. Es gehört zur Internetprotokollfamilie erklärt durch TCP/IP und wird der TCP/IP Schicht für Internet bzw. der OSI-Schicht 3 (Netzwerk/ Vermittlung) zugeordnet. Es ist ein Protokoll zu Wegfingung in einem Netzwerk und bietet hierarchische Adressen.

IPv4 benutzt 32-Bit-Adressen. Eine Häufige Darstellungsart ist die Dezimalpunktschreibweise (decimal dotted), welche die Adresse in 4 Bytes teilt und diese durch Punkte getrennt dezimal darstellt. Diese Schreibweise ist menschlich besser lesbar aber sperrig bei Subnetting, welches innerhalb der sog. Oktette stattfindet. Der IPv4-Adressraum bietet in einem Gesamtnetz Platz für 4,3 Milliarden Adressen.