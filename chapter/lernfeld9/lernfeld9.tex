\chapter{Lernfeld 9: Netzwerke und Dienste bereitstellen}

\textbf{Die Schülerinnen und Schüler verfügen über die Kompetenz, Netzwerke und Dienste
    zu planen, zu konfigurieren und zu erweitern.}

Die Schülerinnen und Schüler ermitteln die Anforderungen an ein Netzwerk in Kommunikation mit den Kunden. Sie \textbf{informieren} sich über Eigenschaften, Funktionen und Leistungsmerkmale der Netzwerkkomponenten und Dienste nach Kundenanforderung, auch unter
Berücksichtigung sicherheitsrelevanter Merkmale. Dabei wenden sie Recherchemethoden
an und werten auch fremdsprachliche Quellen aus.

Sie \textbf{planen} die erforderlichen Dienste und dafür notwendige Netzwerke sowie deren Infrastruktur unter Berücksichtigung interner und externer Ressourcen.

Dazu \textbf{vergleichen} sie Konzepte hinsichtlich ihrer Nachhaltigkeit sowie der technischen und
wirtschaftlichen Eignung.

Sie \textbf{installieren} und konfigurieren Netzwerke sowie deren Infrastruktur und implementieren
Dienste. Sie gewährleisten die Einhaltung von Standards, führen Funktionsprüfungen sowie
Messungen durch und erstellen eine Dokumentation.

Die Schülerinnen und Schüler \textbf{beurteilen} die Netzwerke sowie deren Infrastruktur und die
Dienste hinsichtlich der gestellten Anforderungen, Datensicherheit und Datenschutz.

Sie \textbf{reflektieren} ihre Lösung unter Berücksichtigung der Kundenzufriedenheit, Zukunftsfähigkeit und Vorgehensweise.

\section{IPv4}