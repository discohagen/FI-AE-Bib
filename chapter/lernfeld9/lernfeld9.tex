\chapter{Lernfeld 9: Netzwerke und Dienste bereitstellen}

\textbf{Die Schülerinnen und Schüler verfügen über die Kompetenz, Netzwerke und Dienste
    zu planen, zu konfigurieren und zu erweitern.}

Die Schülerinnen und Schüler ermitteln die Anforderungen an ein Netzwerk in Kommunikation mit den Kunden. Sie \textbf{informieren} sich über Eigenschaften, Funktionen und Leistungsmerkmale der Netzwerkkomponenten und Dienste nach Kundenanforderung, auch unter
Berücksichtigung sicherheitsrelevanter Merkmale. Dabei wenden sie Recherchemethoden
an und werten auch fremdsprachliche Quellen aus.

Sie \textbf{planen} die erforderlichen Dienste und dafür notwendige Netzwerke sowie deren Infrastruktur unter Berücksichtigung interner und externer Ressourcen.

Dazu \textbf{vergleichen} sie Konzepte hinsichtlich ihrer Nachhaltigkeit sowie der technischen und
wirtschaftlichen Eignung.

Sie \textbf{installieren} und konfigurieren Netzwerke sowie deren Infrastruktur und implementieren
Dienste. Sie gewährleisten die Einhaltung von Standards, führen Funktionsprüfungen sowie
Messungen durch und erstellen eine Dokumentation.

Die Schülerinnen und Schüler \textbf{beurteilen} die Netzwerke sowie deren Infrastruktur und die
Dienste hinsichtlich der gestellten Anforderungen, Datensicherheit und Datenschutz.

Sie \textbf{reflektieren} ihre Lösung unter Berücksichtigung der Kundenzufriedenheit, Zukunftsfähigkeit und Vorgehensweise.

Die folgenden Informationen beschäftigen sich hauptsächlich mit Netzwerksoftware. Für Netzwerkhardware, Struktur und Physik siehe Lernfeld 3.

\section{Netzwerksoftwaremodelle}

Netzwerke sind meist in hierarchische Schichten (Layers) aufgeteilt. Jede Schicht bietet den höheren Schichten bestimmte Services und kümmert sich alleine um die Implementation dieser Services.

Die Kommunikation findet dabei in der Theorie in den einzigen Schichten statt und wird von Protokollen dieser Schichten geregelt.

\begin{figure}[H]
    \centering
    \includesvg[width=\textwidth,inkscapelatex=false]{figures/Netzwerkmodell.drawio.svg}
    \caption{Beispiel Schichtmodell allgemein}
    \label{fig:schichtmodell}
\end{figure}
\FloatBarrier

Ein Host hat pro Schicht eine Kommunikationskomponente gen. Peer, die mit einem Peer der gleichen Schicht eines anderen Hosts kommunizieren kann. Praktisch sind Peers z.B. Softwareprozesse, Hardwaregeräte oder Menschen.

Praktisch findet die Kommunikation nicht horizontal innerhalb der einzelnen Schichten, sonder vertikal zwischen den Schichten statt. Jede Schicht fügt Informationen hinzu, die für ihr Protokoll notwendig sind. Die Übertragung findet auf der untersten Schicht in einem physischen Medium statt.

Zwischen zwei Schichten definieren Schnittstellen (Interfaces) ähnlich Blaupausen welche Services die untere Schicht anbieten sollte.

Eine Menge an Schichten und Protokollen wird als Netzwerkarchitektur bezeichnet. Eine Liste an Protokollen mit einem Protokoll für jede Schicht wird als Protokollstack bezeichnet.

Protokolle benötigen mindestens Steuerinformationen zusätzlich zu den eigentlichen Nutzdaten. Jedes Protokoll fügt Informationen hinzu und gibt seine Nutzdaten plus die zusätzliche Informationen an das untere Protokoll als seine Nutzdaten weiter. Dieser Vorgang läuft auf der Zielmaschine Rückwärts ab. Das zugrunde liegende Prinzip nennt sich Datenkapselung.

\begin{figure}[H]
    \centering
    \includesvg[width=\textwidth,inkscapelatex=false]{figures/Datenkapselung.drawio.svg}
    \caption{Beispiel Datenkapselung im Netzwerkmodell}
    \label{fig:Datenkapselung}
\end{figure}
\FloatBarrier

\subsection{ISO Open Systems Interconnection Referenzmodell (OSI-Modell)}

Das OSI-Modell ist keine Netzwerkarchitektur aber eine Vorlage für diese. Die einzelnen Schichten sind nach Funktionen getrennt.

\begin{table}[H]
    \centering
    \begin{tabular}{|l|l|}
        \hline
        7 & Application/ Anwendung    \\\hline
        6 & Presentation/ Darstellung \\\hline
        5 & Session/ Sitzung          \\\hline
        4 & Transport                 \\\hline
        3 & Network/ Vermittlung      \\\hline
        2 & Data Link/ Sicherung      \\\hline
        1 & Physical/ Bitübertragung  \\\hline
    \end{tabular}
    \caption{ISO OSI-Modell Übersicht}
\end{table}

\begin{figure}[H]
    \centering
    \includesvg[width=\textwidth,inkscapelatex=false]{figures/OSI_allgemein.drawio.svg}
    \caption{OSI-Modell}
\end{figure}
\FloatBarrier

\begin{table}
    [H]
    \centering
    \begin{tabularx}{\textwidth}{|c|l|l|>{\raggedright\arraybackslash}X|>{\raggedright\arraybackslash}X|>{\raggedright\arraybackslash}X|>{\raggedright\arraybackslash}X|}
        \hline
        \textbf{ID} & \textbf{OSI}                     & \textbf{TCP/IP} & \textbf{Protokolle u. Techniken}  & \textbf{Einheiten} & \textbf{Kopplung}            & \textbf{Verbindung} \\\hline
        7           & Application                      & Application     & DHCP, DNS, FTP, HTTP, HTTPS, SMTP & Daten              & Gateway, Proxy               & Ende zu Ende        \\\cline{1-2}
        6           & Presentation                     &                 &                                   &                    &                              &                     \\\cline{1-2}
        5           & Session                          &                 &                                   &                    &                              &                     \\\cline{1-5}
        4           & \multicolumn{2}{c|}{Transport  } & TCP, UDP        & Segmente, Datagramme              &                    &                                                    \\\cline{1-6}
        3           & Network                          & Internet        & ICMP, IP                          & Pakete             & Router                       &                     \\\cline{1-7}
        2           & Data Link                        & Link            & Ethernet, WLAN, MAC               & Frames             & Bridge, Switch, WAP          & Punkt zu Punkt      \\\cline{1-2}\cline{5-6}
        1           & Physical                         &                 &                                   & Bits, Symbole      & Netzwerkkabel, Repeater, Hub &                     \\\hline
    \end{tabularx}
    \caption{Vollständige Übersicht OSI und TCP/IP}
\end{table}

\subsection{Transmission Control Protocol/ Internet Protocol (TCP/IP)}

Der TCP/IP-Stack ist die Gruppierung von Netzwerkprotokollen. Im weiteren Sinne der gesamte Stack für die Internetprotokollfamilie.

Der Stack wird u.a. im TCP/IP-Referenzmodell dargestellt. Das Modell gilt damit als darstellung der TCP/IP-Netzwerkarchitektur.

\begin{table}
    [H]
    \centering
    \begin{tabular}{|l|l|l|}
        \hline
        \multicolumn{2}{|l|}{\textbf{Schicht}} & \textbf{Besipielprotokolle}                                      \\\hline
        4                                      & Application/ Anwendung      & HTTP, UDP, FTP, SMTP, Telnet, DHCP \\\hline
        3                                      & Transport                   & TCP, UDP                           \\\hline
        2                                      & Internet                    & IP, ICMP                           \\\hline
        1                                      & Link/ Netzzugang            & \multicolumn{1}{c}{}               \\\cline{1-2}
    \end{tabular}
    \caption{TCP/IP-Modell Übersicht}
\end{table}

Die \textbf{Anwendungsschicht} befasst sich hier mit Protokollen für Anwendungsprogramme und die Netzinfrakstruktur für anwendungsspezifischen Datenaustausch.

Die \textbf{Transportschicht} befasst sich mit der Ende-zu-Ende-Kommunikation.

Die \textbf{Internetschicht} befasst sich mit der Übertragung von Paketen und der Wegwahl (Routing) dieser. Auf dieser Schicht und der Netzzugangsschicht werden Direktverbindungen betrachtet.

Die \textbf{Netzzugangsschicht} beinhaltet keine Protokolle per se sondern ist ein Platzhalter für Techniken der Datenübertragung von Punkt zu Punkt.

\section{TCP/IP-Stack Protokolle der Anwendungsschicht}

\subsection{DHCP - Dynamic Host Configuration Protocol}

DHCP ermöglicht die automatische Zuweisung von Netzwerkkonfigurationen (wie die IP-Adresse) an Clients durch einen Server im Gegensatz zu einer manuellen Konfiguration der CLients.

\subsection{DNS - Domain Name System}

DNS löst Domainnamen zu IP-Adressen auf.

\subsection{FTP(S) - File Transfer Protocol (Secure)}

FTP ist eine zustandsbehaftete Möglichkeit Dateien vom Client auf einen Server hochzuladen, herunterzuladen oder clientgesteuert Dateien zwischen zwei Servern zu übertragen. Es kann außerdem genutzt werden um Verzeichnisse anzulegen und diese und Dateien umzubenenen und zu löschen.

FTPS ist die Secure Version von FTP via TLS.

\subsection{HTTP(S) - Hypertext Transfer Protocol (Secure)}

HTTP ist eine zustandslose Möglichkeit Daten zu übertragen. Hauptsächlich werden Hypertext-Dokumente aus dem World Wide Web (WWW) in ein Webbrowser geladen.

HTTPS ist die Secure Version von HTTP und baut auf TLS auf.

\subsection{IMAP - Internet Message Access Protocol}

IMAP (ursprünglich Interactive Message Access Protocol) stellt ein Netzwerkdateisystem für E-Mails bereit. Es erweitert die Funktionen von POP um Mails, Ordnerstrukturen und Einstellungen auf Mailservern zu speichern, um die Abhängigkeit von einzelnen Clients zu umgehen.

\subsection{POP - Post Office Protocol}

POP (mittlerweile POP3) erlaubt das Auflisten, Abholen und Löschen von E-Mails von einem E-Mail-Server.

\subsection{SMTP - Simple Mail Transfer Protocol}

SMTP dient dem Versenden von E-Mails.

\subsection{SOCKS - Internet-Sockets Protokoll}

SOCKS erlaubt Client-Server-Anwendungen protokollunabhängig und transparent Dienste eines Proxyservers zu nutzen. Clients hinter einer Firewall verbinden sich nicht direkt mit einem externen Server, sondern zu einem SOCKS-Proxy. Dieser Proxyserver überprüft die Berechtigung des CLients, den externen Server zu kontaktieren und leitet die Anfrage an den Server weiter.

\subsection{SSH - Secure Shell, SCP und SFTP}

SSH ermöglicht den kryptographischen und damit sicheren Betrieb von Netzwerkdiensten über ungesicherte Netzwerke. SSH ist damit eine sichere Alternative zu Telnet. SSH wird auch oftmals genutzt um lokal auf eine entfernete Kommandozeile zuzugreifen.

Secure Copy Protocol (SCP) ist die Dateiübertragung via SSH.

SSH File Transfer Protocol oder Secure File Transfer Protocol (SFTP) ist die zweite Version von SCP.

\section{TCP/IP-Stack Protokolle der Transportschicht}

\subsection{TCP - Transmission Control Protocol}

TCP dient der Verbindung von Netzwerkkomponenten zum Datenaustausch. Im Gegensatz zu UDP stellt TCP eine Verbindung zwischen zwei Endpunkten einer Netzverbindung (gen. Sockets) her. Auf dieser Verbindung können dann Daten in beide Richtungen übertragen werden (allerdings nicht gleichzeitig). Ein Endpunkt wird über die IP-Adresse und einen Port definiert. Eine TCP-Verbindung wird also aus den vier Werten Quell-IP, Quell-Port, Ziel-IP, Ziel-Port identifiziert. Die Identifikation ist über alle Werte eindeutig. Dies bedeutet wiederrum das mehrere eindeutige TCP-Verbindungen zwischen der gleichen Quell-IP mit gleichem Quell-Port zu einer Ziel-IP auf unterschiedlichen Ports möglich ist.

\begin{table}
    [H]
    \begin{bytefield}[bitwidth={\textwidth/32}]{32}
        \bitheader{0-31} \\
        \bitbox{16}{source port} & \bitbox{16}{destination port} \\
        \bitbox{32}{sequence number} \\
        \bitbox{32}{acknowledgment number} \\
        \bitbox{4}{data offset} & \bitbox{4}{reserved} & \bitbox{8}{control flags} & \bitbox{16}{window} \\
        \bitbox{16}{checksum} & \bitbox{16}{urgent pointer} \\
        \wordbox[lrt]{1}{options $\le 40B$} \\
        \wordbox[lrb]{1}{$\cdots$} \\
        \wordbox[lrt]{1}{payload} \\
        \wordbox[lrb]{1}{$\cdots$} \\
    \end{bytefield}
    \caption{TCP-Segment}
\end{table}

\subsection{UDP - User Datagram Protocol}

UDP ist eine minimale und verbindungslose alternative zu TCP. Außerdem ist es unzuverläßig, ungesichert und ungeschützt, so dass Pakete mögl. nicht, in falscher Reihenfolgen, mehr als einmal, manipuliert und oder einsehbar für Dritte ankommen.

\begin{table}
    [H]
    \begin{bytefield}[bitwidth={\textwidth/32}]{32}
        \bitheader{0-31} \\
        \bitbox{16}{source port} & \bitbox{16}{destination port} \\
        \bitbox{16}{total lenght} & \bitbox{16}{checksum} \\
        \wordbox[lrt]{1}{payload} \\
        \wordbox[lrb]{1}{$\cdots$} \\
    \end{bytefield}
    \caption{UDP-Datagramm}
\end{table}

\subsection{TLS - Transport Layer Securit}

TLS (ehemals SSL (Secure Socket Layer)) ist ein Verschlüsselungsprotokoll.

\section{TCP/IP-Stack Protokolle der Internetschicht}

\subsection{IP - Internet Protocol Grundlagen}

IP ist ein verbindungsloses Protokoll.

\begin{table}
    [H]
    \begin{bytefield}[bitwidth={\textwidth/32}]{32}
        \bitheader{0-31} \\
        \bitbox{4}{version} & \bitbox{4}{IHL} & \bitbox{8}{TOS} & \bitbox{16}{total length} \\
        \bitbox{16}{identification} & \bitbox{3}{flags} & \bitbox{13}{fragment offset} \\
        \bitbox{8}{TTL} & \bitbox{8}{protocol} & \bitbox{16}{header checksum} \\
        \bitbox{32}{source ipv4} \\
        \bitbox{32}{destination ipv4} \\
        \wordbox[lrt]{1}{options and padding $\le$ 40 B} \\
        \wordbox[lrb]{1}{$\cdots$} \\
        \wordbox[lrt]{1}{payload} \\
        \wordbox[lrb]{1}{$\cdots$} \\
    \end{bytefield}
    \caption{IPv4-Paket}
\end{table}

\begin{itemize}
    \item IHL (Internet Header Length): Gesamtlänge des Headers in Vielfachen von 32 Bit
    \item TOS (Type of Service): Priorisierung von IP-Paketen (Quality of Service)
    \item flags: Bit 0 ist reserviert; Bit 1 zeigt an ob das Paket fragmentiert werden darf (Don't fragment); Bit 2 zeigt an ob mehr Fragmente folgen (More fragments)
    \item TTL (Time to Live): Ursprünglich Lebenszeit in Sekunden - heute Hop-Counter, welcher bei jedem Router runtergezählt wird; Bei 0 wird das Paket verworfen
    \item protocol: Das Protokoll, welches für die Nutzdaten verwendet wird (6: TCP; 17: UDP)
\end{itemize}

\begin{table}
    [H]
    \begin{bytefield}[bitwidth={\textwidth/32}]{32}
        \bitheader{0-31} \\
        \bitbox{4}{version} & \bitbox{8}{traffic class} & \bitbox{20}{flow label} \\
        \bitbox{16}{payload length} & \bitbox{8}{next header} & \bitbox{8}{hop limit} \\
        \wordbox{4}{source ipv6} \\
        \wordbox{4}{destination ipv6} \\
    \end{bytefield}
    \caption{IPv6-Paket}
\end{table}

\begin{itemize}
    \item traffic class: Quality of Service
    \item flow label: Ebenfalls für Quality of Service oder Echtzeitanwendungen verwendeter WertK; Pakete mit gleichem flow label werden gleich behandelt
    \item payload length: Länge der Nutzdaten inkl. Erweiterungs-Kopfdaten
    \item next header: nächster extension header oder Protokoll für Nutzdaten (6: TCP; 17: UDP)
\end{itemize}

\begin{table}
    [H]
    \centering
    \begin{tabular}{|l|c|l|}
        \hline
        \textbf{Name}                  & \textbf{Typ} & \textbf{Beschreibung}                                 \\\hline
        Hop-By-Hop Options             & 0            & Optionen für jedes Gerät, welche das Paket durchläuft \\\hline
        Routing                        & 43           & Beeinflussung des Routingweges (z.B. für Mobile IPv6) \\\hline
        Fragment                       & 44           & Parameter zur Fragmentierung (64 Bit)                 \\\hline
        Encapsulation Security Payload & 50           & Verschlüsselungsdaten (IPsec)                         \\\hline
        Authentication Header          & 51           & Authentifizierung des Paketes (IPsec)                 \\\hline
        No Next Header                 & 59           & Ende des Header-Stapels                               \\\hline
        Destination Options            & 60           & Optionen, welche der Zielrechner beachten muss        \\\hline
        Mobility                       & 135          & Daten für Mobile IPv6                                 \\\hline
    \end{tabular}
    \caption{Mögliche Verweise des Next Header Feldes}
\end{table}

\section{TCP/IP-Stack Protokolle der Netzzugangsschicht}

\subsection{ARP - Address Resolution Protocol}

ARP soll zu einer IP-Adresse die zugehörige MAC-Adresse ermitteln. Es kann via Broadcast oder Unicast nach der MAC-Adresse einer bestimmten IP gefragt werden (ARP-Request) oder es kann eine bestimmte Zuordnung beliebigen Teilnehmern mitgeteilt werden (ARP-Reply). Jeder Client hält außerdem einen ARP-Cache, welcher durch ARP-Replys oder beliebige Pakete, welche IP-MAC-Zuordnungen beinhalten, aktualisiert werden kann. ARP ist an sich generisch designt, wird aber in der Praxis nur für IPv4 genutzt (für IPv6 gibt es das Neighbor Discovery Protocol).

\begin{table}
    [H]
    \begin{bytefield}[bitwidth={\textwidth/32}]{32}
        \bitheader{0-31} \\
        \bitbox{16}{hardware address type} \bitbox{16}{protocol address type} \\
        \bitbox{8}{hardware address size} \bitbox{8}{protocol address size} \bitbox{16}{operation} \\
        \bitbox[ltb]{32}{source mac $\cdots$} \\
        \bitbox[rtb]{16}{$\cdots$ source mac} \bitbox[ltb]{16}{source ip $\cdots$} \\
        \bitbox[rtb]{16}{$\cdots$ source ip} \bitbox[ltb]{16}{destination mac $\cdots$} \\
        \bitbox[rtb]{32}{$\cdots$ destination mac} \\
        \bitbox{32}{destination ip} \\
    \end{bytefield}
    \caption{ARP-Paket}
\end{table}

\subsection{Ethernet und WLAN}

Ethernet an sich ist eine Technik, welche Soft- und Hardware für kabelgebundene Datennetze spezifiziert. WLAN ist das Pendant für kabellose Datennetze. Es gibt verschieden Standards und Protokolle innerhalb beider Techniken. Die einfachsten und klassischen sind gelistet.

\begin{table}
    [H]
    \centering
    \begin{tabulary}{\textwidth}{|C|C|C|C|C|C|C|C|C|}
        \hline
        OSI & preamble               & SFD  & destination mac & source mac & type     & payload & FCS   & IPG \\\hline
        2   & \multicolumn{2}{c|}{-} & $6B$ & $6B$                    & $2B$               & $\cdots$ & $4B$    &   -          \\\cline{1-3}\cline{9-9}
        1   & $7B$                   & $1B$ &                     &                    &          &       &  & $12B$       \\\hline
    \end{tabulary}
    \caption{Ethernet-II}
\end{table}

\begin{itemize}
    \item SFD: Start frame delimiter
    \item FCS: Frame check sequence
    \item IPG: Internetpacket gap
\end{itemize}

\begin{table}
    [H]
    \centering
    \begin{tabulary}{\textwidth}{|C|C|C|C|C|C|C|C|C|C|C|}
        \hline
        frame control & duration/ id & mac 1 & mac 2 & mac 3 & sequence control & mac 4 & payload  & FCS  \\\hline
        $2B$          & $2B$         & $6B$  & $6B$  & $6B$  & $2B$             & $6B$  & $\cdots$ & $4B$ \\\hline
    \end{tabulary}
    \caption{IEEE 802.11 (WLAN)}
\end{table}

\begin{itemize}
    \item frame control: Typ des Frames und wie dieser verarbeitet wird
    \item Für Daten- und Kontroll-Frames Dauer, wie lange das Medium benutzt wird; für Magament-Frames optionale ID
    \item mac 1: destination mac
    \item mac 2: source mac
    \item mac 3: Im Infrastruktur-Modus gateway mac; im Ad-hoc-Modus (P2P) destination peer; im WDS (Wireless Distribution System) next chain mac
    \item mac 4: Im WDS source gateway mac
\end{itemize}

\section{IPv4}

\section{IPv6}