\chapter{Lernfeld 4: Schutzbedarfsanalyse im eigenen Arbeitsbereich durchführen}

\textbf{Die Schülerinnen und Schüler verfügen über die Kompetenz, mit Hilfe einer bestehenden Sicherheitsleitlinie eine Schutzbedarfsanalyse zur Ermittlung der Informationssicherheit auf Grundschutzniveau in ihrem Arbeitsbereich durchzuführen.}

Die Schülerinnen und Schüler \textbf{informieren} sich über Informationssicherheit (Schutzziele)
und rechtliche Regelungen sowie die Einhaltung von betrieblichen Vorgaben zur Bestimmung des Schutzniveaus für den eigenen Arbeitsbereich.

Sie \textbf{planen} eine Schutzbedarfsanalyse, indem sie gemäß der IT-Sicherheitsleitlinie des Unternehmens Schutzziele des Grundschutzes (Vertraulichkeit, Integrität, Verfügbarkeit) in ihrem Arbeitsbereich ermitteln und eine Klassifikation von Schadensszenarien vornehmen.

Sie \textbf{entscheiden} über die Gewichtung möglicher Bedrohungen unter Berücksichtigung der
Schadenszenarien.

Dazu \textbf{führen} sie eine Schutzbedarfsanalyse in ihrem Arbeitsbereich \textbf{durch}, nehmen Bedrohungsfaktoren auf und dokumentieren diese.


Die Schülerinnen und Schüler \textbf{bewerten} die Ergebnisse der Schutzbedarfsanalyse und gleichen diese mit der IT-Sicherheitsleitlinie des Unternehmens ab. Sie empfehlen Maßnahmen
und setzen diese im eigenen Verantwortungsbereich um.

Sie \textbf{reflektieren} den Arbeitsablauf und übernehmen Verantwortung im IT-Sicherheitsprozess.

\section{BSI (Bundesamt für Sicherheit in der Informationstechnik)}

Das BSI ist eine Deutsche Behörde die Informationen über IT-Sicherheit gesammelt darstellt und für den deutschen Raum bestimmt.

\section{Schutzziele}

Allgemein kann IT-Sicherheit als einhalten der Informationssicherheit und damit dem Schutz von Daten vor Angriffen in drei hauptsächliche Schutzziele und mehrere weitere Schutzziele unterteilt werden.

\textbf{Vertraulichkeit (Confidentiality)}: Zugriff auf Daten haben nur bestimmte und berechtigte Personen und Gruppen.

\textbf{Integrität (Integrity)}: Daten sind nicht manipuliert oder unbemerkt oder nicht nachvollziehbar verändert.

\textbf{Verfügbarkeit (Availability)}: Daten sind in bestimmten Zeiten verfügbar.

\begin{itemize}
    \item Authentizität (Authenticity): Echtheit, Überprüfbarkeit und Vertrauenswürdigkeit
    \item Verbindlichkeit (Non-Repudiation): Nachweisbarkeit bzw. Nichtabstreitbarkeit
    \item Zurechenbarkeit (Accountability): Eindeutige Zuordnung
    \item Anonymität (Anonymity): Keine Zuordnung oder Nachweisbarkeit bei kritischen und oder bestimmten Daten
    \item Resilienz (Resilience): Widerstandsfähigkeit gegenüber Angriffen und irrtümlichen, mutwilligen oder absichtlichen Störungen oder Schädigungen
\end{itemize}

\input{chapter/lernfeld4/sections/Cyberangriffe.tex}

\input{chapter/lernfeld4/sections/Verschlüsselung.tex}

