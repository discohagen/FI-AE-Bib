\chapter{Lernfeld 4: Schutzbedarfsanalyse im eigenen Arbeitsbereich durchführen}

\textbf{Die Schülerinnen und Schüler verfügen über die Kompetenz, mit Hilfe einer bestehenden Sicherheitsleitlinie eine Schutzbedarfsanalyse zur Ermittlung der Informationssicherheit auf Grundschutzniveau in ihrem Arbeitsbereich durchzuführen.}

Die Schülerinnen und Schüler \textbf{informieren} sich über Informationssicherheit (Schutzziele)
und rechtliche Regelungen sowie die Einhaltung von betrieblichen Vorgaben zur Bestimmung des Schutzniveaus für den eigenen Arbeitsbereich.

Sie \textbf{planen} eine Schutzbedarfsanalyse, indem sie gemäß der IT-Sicherheitsleitlinie des Unternehmens Schutzziele des Grundschutzes (Vertraulichkeit, Integrität, Verfügbarkeit) in ihrem Arbeitsbereich ermitteln und eine Klassifikation von Schadensszenarien vornehmen.

Sie \textbf{entscheiden} über die Gewichtung möglicher Bedrohungen unter Berücksichtigung der
Schadenszenarien.

Dazu \textbf{führen} sie eine Schutzbedarfsanalyse in ihrem Arbeitsbereich \textbf{durch}, nehmen Bedrohungsfaktoren auf und dokumentieren diese.


Die Schülerinnen und Schüler \textbf{bewerten} die Ergebnisse der Schutzbedarfsanalyse und gleichen diese mit der IT-Sicherheitsleitlinie des Unternehmens ab. Sie empfehlen Maßnahmen
und setzen diese im eigenen Verantwortungsbereich um.

Sie \textbf{reflektieren} den Arbeitsablauf und übernehmen Verantwortung im IT-Sicherheitsprozess.

% Datensicherheit allgemein
% CIA, Vertraulichkeit, Integrität, Verfügbarkeit
% symmetrische und asymmetrische Verschlüsselung
% digitale Signatur
% Hashwertverfahren und Hashfunktionen
