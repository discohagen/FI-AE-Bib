\chapter{Lernfeld 4: Schutzbedarfsanalyse im eigenen Arbeitsbereich durchführen}

\textbf{Die Schülerinnen und Schüler verfügen über die Kompetenz, mit Hilfe einer bestehenden Sicherheitsleitlinie eine Schutzbedarfsanalyse zur Ermittlung der Informationssicherheit auf Grundschutzniveau in ihrem Arbeitsbereich durchzuführen.}

Die Schülerinnen und Schüler \textbf{informieren} sich über Informationssicherheit (Schutzziele)
und rechtliche Regelungen sowie die Einhaltung von betrieblichen Vorgaben zur Bestimmung des Schutzniveaus für den eigenen Arbeitsbereich.

Sie \textbf{planen} eine Schutzbedarfsanalyse, indem sie gemäß der IT-Sicherheitsleitlinie des Unternehmens Schutzziele des Grundschutzes (Vertraulichkeit, Integrität, Verfügbarkeit) in ihrem Arbeitsbereich ermitteln und eine Klassifikation von Schadensszenarien vornehmen.

Sie \textbf{entscheiden} über die Gewichtung möglicher Bedrohungen unter Berücksichtigung der
Schadenszenarien.

Dazu \textbf{führen} sie eine Schutzbedarfsanalyse in ihrem Arbeitsbereich \textbf{durch}, nehmen Bedrohungsfaktoren auf und dokumentieren diese.


Die Schülerinnen und Schüler \textbf{bewerten} die Ergebnisse der Schutzbedarfsanalyse und gleichen diese mit der IT-Sicherheitsleitlinie des Unternehmens ab. Sie empfehlen Maßnahmen
und setzen diese im eigenen Verantwortungsbereich um.

Sie \textbf{reflektieren} den Arbeitsablauf und übernehmen Verantwortung im IT-Sicherheitsprozess.

\section{BSI (Bundesamt für Sicherheit in der Informationstechnik)}

Das BSI ist eine Deutsche Behörde die Informationen über IT-Sicherheit gesammelt darstellt und für den deutschen Raum bestimmt.

\section{Schutzziele}

Allgemein kann IT-Sicherheit als einhalten der Informationssicherheit und damit dem Schutz von Daten vor Angriffen in drei hauptsächliche Schutzziele und mehrere weitere Schutzziele unterteilt werden.

\textbf{Vertraulichkeit (Confidentiality)}: Zugriff auf Daten haben nur bestimmte und berechtigte Personen und Gruppen.

\textbf{Integrität (Integrity)}: Daten sind nicht manipuliert oder unbemerkt oder nicht nachvollziehbar verändert.

\textbf{Verfügbarkeit (Availability)}: Daten sind in bestimmten Zeiten verfügbar.

\begin{itemize}
    \item Authentizität (Authenticity): Echtheit, Überprüfbarkeit und Vertrauenswürdigkeit
    \item Verbindlichkeit (Non-Repudiation): Nachweisbarkeit bzw. Nichtabstreitbarkeit
    \item Zurechenbarkeit (Accountability): Eindeutige Zuordnung
    \item Anonymität (Anonymity): Keine Zuordnung oder Nachweisbarkeit bei kritischen und oder bestimmten Daten
    \item Resilienz (Resilience): Widerstandsfähigkeit gegenüber Angriffen und irrtümlichen, mutwilligen oder absichtlichen Störungen oder Schädigungen
\end{itemize}

\input{chapter/lernfeld4/sections/Cyberangriffe.tex}

\section{Verschlüsselung}

Verschlüsselung kann man allgemein in zwei Arten unterteilen. Bei beiden Verfahren werden Schlüssel zwischen Kommunikationspartnern getauscht.

Bei der \textbf{symmetrischen Verschlüsselung} existiert ein Schlüssel der vorab sicher zwischen den Kommunikationspartnern ausgetauscht wurde und von diesen sicher verwahrt wird. Der Schlüssel kann von allen Beteiligten zum Ver- und Entschlüsseln von Daten genutzt werden. Dieses Verfahren ist vergleichsweise schnell.

Bei der \textbf{asymmetrischen Verschlüsselung} existiert ein Schlüsselpaar dessen Schlüssel jeweils nur für die Aufgabe des Ver- bzw. Entschlüsseln genutzt werden können. Der \textbf{Public Key} dient der Verschlüsselung und muss nicht geheim gehalten werden. Der \textbf{Private Key} muss geheim gehalten werden und dient der Entschlüsselung. Damit von einem Schlüssel nicht auf den anderen geschlossen werden kann, wird das Hashwertverfahren genutzt. Die asymmetrische Verschlüsselung ist langsamer als die symmetrische Verschlüsselung wegen dem Rechenaufwand durch die Hashwertberechnung.

Das Verfahren für eine \textbf{digitale Signatur} baut auf der asymmetrischen Verschlüsselung auf. Allerdings wird hier der private Key genutzt um Informationen zu verschlüsseln, bzw. eine Signatur zu bilden. Der öffentliche Schlüssel kann dann genutzt werden um die Signatur zu überprüfen, aber nur derjenige mit dem private Key kann die Signatur erstellen und sich so mit dieser authentifizieren.

\subsection{Hashfunktionen}

Eine Hashfunktion ist allgemein eine Einwegfunktion. Sie ist eine Abbildung, welche eine Eingabemenge auf eine kleinere Zielmenge (gen. Hashwerte) abbildet. Mehrere Elemente der Eingabemenge können auf den gleichen Hashwert abgebildet werden. Es wird aber immer die gleiche Eingabe auf den gleichen Hashwert abgebildet. Die Hashwerte haben eine feste Länge, unabhängig von der Länge der Eingabe. 

