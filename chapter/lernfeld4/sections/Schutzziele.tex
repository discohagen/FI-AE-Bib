\section{Schutzziele}

Allgemein kann IT-Sicherheit als einhalten der Informationssicherheit und damit dem Schutz von Daten vor Angriffen in drei hauptsächliche Schutzziele und mehrere weitere Schutzziele unterteilt werden.

\textbf{Vertraulichkeit (Confidentiality)}: Zugriff auf Daten haben nur bestimmte und berechtigte Personen und Gruppen.

\textbf{Integrität (Integrity)}: Daten sind nicht manipuliert oder unbemerkt oder nicht nachvollziehbar verändert.

\textbf{Verfügbarkeit (Availability)}: Daten sind in bestimmten Zeiten verfügbar.

\begin{itemize}
    \item Authentizität (Authenticity): Echtheit, Überprüfbarkeit und Vertrauenswürdigkeit
    \item Verbindlichkeit (Non-Repudiation): Nachweisbarkeit bzw. Nichtabstreitbarkeit
    \item Zurechenbarkeit (Accountability): Eindeutige Zuordnung
    \item Anonymität (Anonymity): Keine Zuordnung oder Nachweisbarkeit bei kritischen und oder bestimmten Daten
    \item Resilienz (Resilience): Widerstandsfähigkeit gegenüber Angriffen und irrtümlichen, mutwilligen oder absichtlichen Störungen oder Schädigungen
\end{itemize}