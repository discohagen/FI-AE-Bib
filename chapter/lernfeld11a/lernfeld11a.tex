\chapter{Lernfeld 11a: Funktionalität in Anwendungen realisieren}

\textbf{Die Schülerinnen und Schüler verfügen über die Kompetenz, modulare Komponenten
    zur informationstechnischen Verarbeitung von Arbeitsabläufen und Geschäftsprozessen zu entwickeln und deren Qualität zu sichern.}

Die Schülerinnen und Schüler \textbf{leiten} aus den Informationsobjekten der vorgegebenen Prozessbeschreibungen der Kunden die dazu notwendigen Datenstrukturen und Funktionalitäten \textbf{ab}.

Sie \textbf{planen} modulare Softwarekomponenten und beschreiben deren Funktionsweise mit Diagrammen und Modellen.

Sie \textbf{wählen} eine Methode zur Softwareentwicklung \textbf{aus}. Dabei beachten sie, dass Planung,
Realisierung und Tests iterativ in Abstimmung mit den Kunden erfolgen.

Die Schülerinnen und Schüler \textbf{realisieren}, auch im Team, die Softwarekomponenten und
binden diese an Datenquellen an. Sie dokumentieren die Schnittstellen.

Sie \textbf{testen} die erforderliche Funktionalität, indem sie Testfälle formulieren und automatisierte Testverfahren anwenden.

Die Schülerinnen und Schüler \textbf{beurteilen} die Funktionalität anhand festgelegter Kriterien
der Kunden und leiten Maßnahmen zur Überarbeitung der erstellten Module ein.