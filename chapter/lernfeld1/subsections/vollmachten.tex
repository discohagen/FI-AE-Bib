\subsection{Vollmachten und Prokura}

Prokura ist eine Vertretungsmacht zur Geschäftsführung deren Umfang gesetzlich geregelt ist. Handlungsvollmachten begrenzen sich dagegen auf bestimmte Geschäfte.

Handlungsvollmacht ist dabei ein Oberbegriff für Generalhandlungsvollmachten und allgemeine Vollmachten, welche zum Führen des täglichen Geschäftes ermächtigen, Artvollmachten, welche sich auf einen finanziellen Rahmen oder auf einen bestimmten Handlungsbereich beschränken und Sondervollmachten, welche einmalig für explizite Geschäfte erteilt werden.