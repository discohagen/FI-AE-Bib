\section{Rechtsformen}

Rechtsformen beziehen sich auf Unternehmen. Unternehmen sind dabei von Betrieben und Firmen folgendermaßen abzugrenzen. Unternehmen sind rechtlich selbstständige organisatorische Einheiten der Volkswirtschaft. Betriebe sind technisch-soziale Einheiten und Unternehmen unterzuordnen. Betriebe beschreiben oft örtliche Einheiten eines Unternehmens. Firmen sind im Handelsregister eingetragene Namen eines Unternehmens. Sie bestehen aus Firmenkern (eigentlicher Name) und Firmenzusatz (Rechtsform).

Der Firmenkern kann sich von Namen (Personenfirmenkern), von Produkten oder Dienstleistungen (Sachfirmenkern), aus einer Mischform (Mischfirmenkern) ableiten oder frei erfunden werden (Fantasie-Firmenkern).

\textbf{Einzelunternehmen}

Der einzelne Unternehmer (eigentragener Kaufmann (E.K.)) hat alleiniges Bestimmungsrecht. Er bringt das gesamte notwendige Kapital auf und erhält den kompletten Gewinn. Er trägt das alleinige Risiko und haftet mit seinem gesamten Betriebs- und Privatvermögen. Es ist die häufigste Rechtsform in Deutschland.

\textbf{Personengesellschaften}

Personengesellschaften werden von mindesten zwei i.d.R. natürlichen Personen gegründet. Zu Personengesellschaften gehören u.a. die Gesellschaft bürgerlichen Rechts (GbR), die offene Handelsgesellschaft (OHG) und die Kommanditgesellschaft (KG). Gesellschafter haften für Gesellschaftsschulden persönlich. Gesellschafter sind Inhaber und meist auch Geschäftsführer.

\textbf{Kapitalgesellschaften}

Kapitalgesellschaften sind z.B. die Gesellschaft mit beschränkter Haftung (GmbH) und die Aktiengesellschaft (AG). Die Haftung ist auf Gesellschaftseinlagen beschränkt. Für die Gründung ist ein Mindestkapital notwendig. Kapitalgesellschaften sind juristische Personen und können von beliebigen Personen geführt werden.

\begin{table}[H]
    \centering
    \begin{tabularx}{\textwidth}{|>{\raggedright\arraybackslash}l|>{\raggedright\arraybackslash}X|>{\raggedright\arraybackslash}X|>{\raggedright\arraybackslash}l|>{\raggedright\arraybackslash}X|>{\raggedright\arraybackslash}X|}
        \hline
             & Mindest\-Gründerzahl              & Haftung                                                                                                        & Mindest\-kapital & Geschäfts\-führung                                                        & Gewinn\-verteilung                                                                                                                                      \\
        \hline
        E.K. & 1                                 & unbeschränkt inkl. Privatvermögen                                                                              & -                & Eingetragener Kaufmann                                                    & Voller Gewinn an den Eingetragenen Kaufmann                                                                                                             \\
        \hline
        GbR  & 2                                 & unbeschränkt inkl. Privatvermögen                                                                              & -                & alle Gesellschafter, sofern im Gesellschaftsvertrag nicht anders geregelt & zu gleichen Teilen auf alle Gesellschafter, sofern im Gesellschaftsvertrag nicht anders geregelt                                                        \\
        \hline
        OHG  & 2                                 & unbeschränkt inkl. Privatvermögen                                                                              & -                & alle Gesellschafter, sofern im Gesellschaftsvertrag nicht anders geregelt & min. 4\% der Einlagen eines Gesellschafters und danach zu gleichen Teilen auf alle Gesellschafter, sofern im Gesellschaftsvertrag nicht anders geregelt \\
        \hline
        KG   & 1 Komplementär und 1 Kommanditist & Betriebs\-vermögen, dann Einlagen der Kommanditisten und zuletzt der Komplementär inkl. seines Privatvermögens & -                & Alle Komplementäre                                                        & min. 4\% der Einlagen eines Gesellschafters und danach oder anstelle davon durch vertragliche Regelungen                                                \\
        \hline
        GmbH & 1                                 & beschränkt auf das Gesellschaftsvermögen                                                                       & 25.000€          & Angestellter Geschäftsführer                                              & Gewinnanteil entsprechend des Kapitalanteils, sofern vertraglich nicht anders geregelt                                                                  \\
        \hline
        AG   & 1                                 & beschränkt auf das Gesellschaftsvermögen                                                                       & 50.000€          & Vorstand                                                                  & Gewinnanteil entsprechend des Aktienanteil oder vertraglich geregelt z.B. mit Dividenden                                                                \\
        \hline
    \end{tabularx}
    \caption{Rechtsformen}
    \label{tab:rechtsformen}
\end{table}

Es gibt außerdem Mischformen wie die GmbH \& Co. KG, bei welcher eine KG von u.a. einer GmbH gegründet wird.