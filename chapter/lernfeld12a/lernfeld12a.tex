\chapter{Lernfeld 12a: Kundenspezifische Anwendungsentwicklung durchführen}

\textbf{Die Schülerinnen und Schüler verfügen über die Kompetenz, einen Kundenauftrag
zur Anwendungsentwicklung vollständig durchzuführen und zu bewerten.}

Die Schülerinnen und Schüler \textbf{führen} in Zusammenarbeit mit den Kunden eine Anforderungsanalyse \textbf{durch} und leiten daraus Projektziele, Anforderungen, gewünschte Ergebnisse, Schulungsbedarfe und Rahmenbedingungen ab.

Auf dieser Basis \textbf{planen} und kalkulieren sie ein Projekt mit den dazugehörigen personellen
und technischen Ressourcen.

Die Schülerinnen und Schüler entwickeln Lösungsvarianten, vergleichen diese anhand festgelegter Kriterien sowie unter Berücksichtigung von Datenschutz und Datensicherheit. Sie
\textbf{wählen} mit den Kunden die beste Lösung aus. Für den vereinbarten Auftrag erstellen sie
ein Dokument über die zu erbringenden Leistungen und ein Angebot.

Die Schülerinnen und Schüler \textbf{implementieren} die gewünschte Lösung. Dabei nutzen sie
Maßnahmen zur Qualitätssicherung. Sie präsentieren den Kunden das Projektergebnis und
führen eine Schulung durch. Sie übergeben den Kunden das Produkt sowie die Dokumentation.

Die Schülerinnen und Schüler \textbf{bewerten} das Projektergebnis auch hinsichtlich Zielerreichung, Wirtschaftlichkeit, Skalierbarkeit und Verlässlichkeit.

Sie \textbf{reflektieren} die Projektdurchführung und das Projektergebnis auch unter Berücksichtigung der kritisch-konstruktiven Kundenrückmeldungen. 