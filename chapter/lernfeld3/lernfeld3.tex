\chapter{Lernfeld 3: Clients in Netzwerke einbinden}

\textbf{Die Schülerinnen und Schüler verfügen über die Kompetenz, eine Netzwerkinfrastruktur zu analysieren sowie Clients zu integrieren.}

Die Schülerinnen und Schüler \textbf{erfassen} im Kundengespräch die Anforderungen an die Integration von Clients (Soft- und Hardware) in eine bestehende Netzwerkinfrastruktur und
leiten Leistungskriterien ab.

Sie \textbf{planen} die Integration in die bestehende Netzwerkinfrastruktur indem sie ein anforderungsgerechtes Konzept auch unter ökologischen und wirtschaftlichen Gesichtspunkten
(Energieeffizienz) erstellen.

Sie \textbf{führen} auf der Basis der Leistungskriterien die Auswahl von Komponenten \textbf{durch}. Sie
konfigurieren Clients und binden diese in das Netzwerk ein.

Sie \textbf{prüfen} systematisch die Funktion der konfigurierten Clients im Netzwerk und protokollieren das Ergebnis.

Sie \textbf{reflektieren} den Arbeitsprozess hinsichtlich möglicher Optimierungen und diskutieren
das Ergebnis in Bezug auf Wirtschaftlichkeit und Ökologie.

Die folgenden Informationen beschäftigen sich hauptsächlich mit Netzwerkhardware, Struktur und Physik. Für Netzwerksoftware (IPv4, IPv6) siehe Lernfeld 9.