\chapter{Lernfeld 2: Arbeitsplätze nach Kundenwunsch ausstatten}

\textbf{Die Schülerinnen und Schüler verfügen über die Kompetenz, die Ausstattung eines
    Arbeitsplatzes nach Kundenwunsch zu dimensionieren, anzubieten, zu beschaffen
    und den Arbeitsplatz an die Kunden zu übergeben.}

Die Schülerinnen und Schüler nehmen den Kundenwunsch für die Ausstattung eines Arbeitsplatzes von internen und externen Kunden entgegen und \textbf{ermitteln} die sich daraus
ergebenden Anforderungen an Soft- und Hardware. Aus den dokumentierten Anforderungen leiten sie Auswahlkriterien für die Beschaffung ab. Sie berücksichtigen dabei die Einhaltung von Normen und Vorschriften (Zertifikate, Kennzeichnung) für den Betrieb und die
Sicherheit von elektrischen Geräten und Komponenten.

Sie \textbf{vergleichen} die technischen Merkmale relevanter Produkte anhand von Datenblättern
und Produktbeschreibungen zur Vorbereitung einer Auswahlentscheidung (Nutzwertanalyse). Dabei beachten sie insbesondere informationstechnische und energietechnische
Kenngrößen sowie Aspekte der Ergonomie und der Nachhaltigkeit (Umweltschutz, Recycling). Sie wenden Recherchemethoden an und werten auch fremdsprachliche Quellen aus.

Sie ermitteln die Energieeffizienz unterschiedlicher Arbeitsplatzvarianten und dokumentieren diese.

Sie vergleichen mögliche Bezugsquellen (quantitativer und qualitativer Angebotsvergleich)
und \textbf{bestimmen} den Lieferanten.

Auf Basis der ausgewählten Produkte und Lieferanten \textbf{erstellen} sie mit vorgegebenen Zuschlagssätzen ein Angebot für die Kunden.

Sie schließen den Kaufvertrag ab und organisieren den Beschaffungsprozess unter Berücksichtigung von Lieferzeiten. Sie nehmen die bestellten Komponenten in Empfang und dokumentieren dabei festgestellte Mängel.

Sie bereiten die Übergabe der beschafften Produkte vor, integrieren IT-Komponenten, konfigurieren diese und nehmen sie unter Berücksichtigung der Arbeitssicherheit in Betrieb. Sie
übergeben den Arbeitsplatz an die Kunden und erstellen ein Übergabeprotokoll.

Sie \textbf{bewerten} die Durchführung des Kundenauftrags und \textbf{reflektieren} ihr Vorgehen. Dabei
berücksichtigen sie die Kundenzufriedenheit und formulieren Verbesserungsvorschläge.

\section{Nutzwertanalyse}
Die Nutzwertanalyse (NWA) ist eine Methode des qualitativen (Angebots-)Vergleich, indem unterschiedlich bestimmte Teilnutzen zu einem Gesamtnutzen addiert werden, welcher gegen Alternativen verglichen werden kann.

\textbf{Vorgehensweise}

\begin{enumerate}
    \item Festlegen der Bewertungskriterien / Teilnutzenaspekten
    \item Festlegen der Gewichtungsfaktoren / Anteile der einzelnen Bewertungskriterien
    \item Aufstellen einer Punkteskala (z.B. Schulnoten, 0-10, 0-100)
    \item Bewertung der Entscheidungsalternativen anhand der Skala
    \item Ermitteln der gewichteten Punktwerte als Produkt aus Bewertung und Gewichtungsfaktor
    \item Summieren der gewichteten Punktwerte
    \item Interpretation der Ergebnisse
\end{enumerate}

\textbf{Beispiel}

\begin{table}[H]
    \centering
    \begin{tabularx}{\textwidth}{|>{\arraybackslash}X|c|>{\centering\arraybackslash}X|>{\centering\arraybackslash}X|>{\centering\arraybackslash}X|>{\centering\arraybackslash}X|>{\centering\arraybackslash}X|>{\centering\arraybackslash}X|}
        \hline
                    &       & \multicolumn{2}{c|}{Unternehmen 1} & \multicolumn{2}{c|}{Unternehmen 2} & \multicolumn{2}{c|}{Unternehmen 3}                                      \\
        \hline
                    & Gew.  & Punkte                             & Gew. Punkte                        & Punkte                             & Gew. Punkte & Punkte & Gew. Punkte \\
        \hline
        Grafikkarte & 20\%  & 3                                  & 60                                 & 2                                  & 40          & 4      & 80          \\
        \hline
        RAM         & 25\%  & 4                                  & 100                                & 3                                  & 75          & 4      & 100         \\
        \hline
        Monitor     & 40\%  & 2                                  & 80                                 & 1                                  & 40          & 4      & 160         \\
        \hline
        Preis       & 15\%  & 3                                  & 45                                 & 4                                  & 60          & 1      & 15          \\
        \hline
                    & 100\% &                                    & 285                                &                                    & 215         &        & 355         \\
        \hline
    \end{tabularx}
    \caption{Beispiel Nutzwertanalyse}
    \label{tab:nutzwertanalyse}
\end{table}

Die Bewertungskriterien bilden das jeweilige Teilnutzen ab. Die Größe eines Nutzens bemisst sich an denjenigen, welche das Gut nutzen, den Zweck, die Situation, der Zeitpunkt und das Gut selbst. Die Bewertungskriterien sollte untereinander nutzenunabhängig sein.

\textbf{Vorteile}

\begin{itemize}
    \item Flexibilität
    \item Schnelligkeit
    \item direkter Vergleich
    \item Eindeutigkeit
\end{itemize}

\textbf{Nachteile}

\begin{itemize}
    \item Subjektivität
    \item Manipulierbarkeit
    \item Ausschluss von Konsequenzen
    \item Niedrige Aussagekraft bei Alternativen mit sehr ähnlichen quantitativen Gesamtnutzen
\end{itemize}

\section{Handelskalkulation}

Die Handelskalkulation wird intern in Netto berechnet. Der Listeneinkaufspreis und Listenverkaufspreis muss für außen also ggf. in Brutto umgerechnet werden.

\subsection{Bezugskalkulation}

\begin{table}[H]
    \centering
    \begin{tabular}{c|l}
          & Listeneinkaufspreis (LEP)       \\
        \hline
        - & Lieferrabatt                    \\
        \hline
        = & Zieleinkaufspreis (ZEP)         \\
        \hline
        - & Lieferskonto                    \\
        \hline
        = & Bareinkaufspreis (BEP)          \\
        \hline
        + & Bezugskosten (z.B Lieferkosten) \\
        \hline
        = & Bezugspreis / Einstandspreis    \\
    \end{tabular}
    \caption{Bezugskalkulation}
\end{table}

\subsection{Quantitativer Angebotsvergleich}

Mithilfe der Bezugskalkulation wird häufig ein quantitativer Angebotsvergleich durchgeführt. Dabei wird der Bezugspreis der verschiedenen Anbieter verglichen, um den günstigsten Anbieter zu ermitteln.

\subsection{Selbstkostenkalkulation im Handel}

\begin{table}[H]
    \centering
    \begin{tabular}{c|l}
          & Bezugspreis     \\
        \hline
        + & Handlungskosten \\
        \hline
        = & Selbstkosten    \\
    \end{tabular}
    \caption{Selbstkostenkalkulation}
\end{table}

Handlungskosten sind alle Kosten im Unternehmen, welche nicht direkt dem Bezug von Ware zugeordnet werden können (z.B Personalkosten, Miete, Steuer). Handlungskosten werden meist prozentual als Handlungskostenzuschlag auf den Bezugspreis aufgeschlagen. Der Handlungskostenzuschlag ist dabei das Verhältnis von Handlungskosten zu Warenaufwänden in einer Periode.

\subsection{Verkaufskalkulation}

\begin{table}
    [H]
    \centering
    \begin{tabular}{c|l}
          & Selbstkosten              \\
        \hline
        + & Gewinn                    \\
        \hline
        = & Barverkaufspreis (BVP)    \\
        \hline
        + & Kundenskonto              \\
        \hline
        = & Zielverkaufspreis (ZVP)   \\
        \hline
        + & Kundenrabatt              \\
        \hline
        = & Listenverkaufspreis (LVP) \\
    \end{tabular}
    \caption{Verkaufskalkulation}
\end{table}

Achtung! Der Kundenskonto bezieht sich auf den Zielverkaufspreis und nicht den Barverkaufspreis. Der Kundenrabatt bezieht sich auf den Listenverkaufspreis und nicht den Zielverkaufspreis.

Es gilt:

\begin{center}
    $ZVP = BVP + \frac{BVP * Kundenskonto}{100\% - Kundenskonto}$
\end{center}

und

\begin{center}
    $LVP = ZVP + \frac{ZVP * Kundenrabatt}{100\% - Kundenrabatt}$
\end{center}

\subsection{Vollständige Handelskalkulation Vorwärts}

Die Vorwärtskalkulation eignet sich für Märkte mit freier Preisgestaltung.

\begin{table}
    [H]
    \centering
    \begin{tabular}{c|l}
          & Listeneinkaufspreis (LEP) \\
        \hline
        - & Lieferrabatt              \\
        \hline
        = & Zieleinkaufspreis (ZEP)   \\
        \hline
        - & Lieferskonto              \\
        \hline
        = & Bareinkaufspreis (BEP)    \\
        \hline
        + & Bezugskosten              \\
        \hline
        = & Bezugspreis               \\
        \hline
        + & Handlungskosten           \\
        \hline
        = & Selbstkosten              \\
        \hline
        + & Gewinn                    \\
        \hline
        = & Barverkaufspreis (BVP)    \\
        \hline
        + & Kundenskonto              \\
        \hline
        = & Zielverkaufspreis (ZVP)   \\
        \hline
        + & Kundenrabatt              \\
        \hline
        = & Listenverkaufspreis (LVP) \\
    \end{tabular}
    \caption{Vollständige Handelskalkulation Vorwärts}
\end{table}

Siehe Verkaufskalkulation für die Berechnung von ZVP und LVP.

\subsection{Rückwärtskalkulation}

Die Rückwärtskalkulation eignet sich für Märkte mit vorgegebenen Verkaufspreisen.

Das Schema der Rückwärtskalkulation ist gleich dem Schema der Vorwärtskalkulation. Allerdings ist letztere Umzudrehen, Vorzeichen werden invertiert und die besondere Berechnung vo ZVP und LVP wie in der Verkaufskalkulation wird stattdessen auf die Berechnung des ZEP und LEP angewendet.

\subsection{Differenzkalkulation}

Die Differenzkalkulation berechnet den Gewinn bei festem oder marktüblichen Listeneinkaufspreis und Listenverkaufspreis als Differenz von Selbstkosten und Barverkaufspreis. Es wird ein beliebiges Schema der Handelskalkulation verwendet und aus beiden Richtung bis zu Selbstkosten und Barverkaufspreis berechnet.

\subsection{Finanzierung}

Finanzierung umfasst alle Maßnahmen der Kapitalbeschaffung. Das Kapital kann dabei in Form von Geld, Gütern oder Wertpapieren zur Verfügung stehen.

Finanzierungsarten können durch die Perspektiven der Herkunft der Mittel und der Rechtsstellung des Kapitalgebers differenziert werden.

\begin{table}
    [H]
    \centering
    \begin{tabularx}{\textwidth}{|l|X|X|X|}
        \cline{3-4}
        \multicolumn{2}{c|}{}            & \multicolumn{2}{c|}{Rechtsstellung des Kapitalgebers}                                                                                                         \\
        \cline{3-4}
        \multicolumn{2}{c|}{}            & Innenfinanzierung                                     & Außenfinanzierung                                                                                     \\
        \hline
        \multirow{3}{*}{Kapitalherkunft} & Eigenfinanzierung                                     & Selbstfinanzierung (aus Gewinn)                               & Einlagen und Beteiligungsfinanzierung \\
        \cline{2-4}
                                         & Fremdfinanzierung                                     & Finanzierung aus Rückstellung                                 & Kreditfinanzierung                    \\
        \cline{2-4}
                                         & Eigen- und Fremdfinanzierung                          & Finanzierung durch Kapitalfreisetzung (Abschreibung, Verkauf) & \multicolumn{1}{c}{}                  \\
        \cline{1-3}
    \end{tabularx}
\end{table}

Selbstfinanzierung ist die Finanzierung aus dem nicht ausgeschütteten Gewinn. Bei Personengesellschaften handelt es sich um Jahresgewinn ohne Privatentnahmen und bei Kapitalgesellschaften um Rücklagen oder Gewinnvortrag.

Einlagen und Beteiligungsfinanzierung wird durch neue Gesellschafter, die Erhöhung der Kapitalanteile (OHG, KG, GmbH), die Erhöhung des Stammkapitals (GmbH) oder die Ausgabe neuer Aktien (AG) erreicht.

Rückstellungen (z.B. für Pensionen) sind Aufwendungen, welch erst in Zukunft zu echten Ausgaben werden. Bis zur Auszahlung kann das zurückgestellte Kapital zu Finanzierungszwecken genutzt werden.

Bei der Außenfinanzierung mit Fremdkapital spricht man von einem Kredit (s.u.).

Finanzierung durch Kapitalfreisetzung ist z.B. durch Abschreibung möglich. Abschreibungen sind Aufwendungen, welche als Bestandteile der erzielten Erlöse wieder in das Unternehmen zurückfließen. Die Abschreibungswerte können bis zum Kauf einer neuen Anlage als zusätzliche Finanzierungsmittel eingesetzt werden.

\textbf{Kredite}

Ein Kredit ist das Überlassen von Geld o.ä. meist gegen Zinsen und der Rückzahlung zu einem bestimmten Zeitpunkt. Der Kreditgeber wird Gläubiger genannt und der Kreditnehmer Schuldner. Außerdem werden i.d.R. Sicherheiten vereinbart, welche der Gläubiger in Anspruch nehmen kann, wenn der Schuldner seinen Pflichten nicht nachkommt.

Es gibt folgende Arten von Krediten.

\textbf{Lieferantenkredit}

Der Lieferer räumt seinem Kunden ein Zahlungsziel zu einem späteren Zeitpunkt ein.

\textbf{Dispositions-/ Kontokorrentkredit}

Kreditinstitute (z.B. Banken) oder auch Lieferer räumen Kunden die Möglichkeit ein, ihr Konto (Girokonto für privat und Kontokorrentkonto für Unternehmen) zu überziehen.

\textbf{Ratenkauf/ Teilzahlung}

Der Rechnungsbetrag wird in Teilen beglichen. Der Käufer erhält die Ware sofort, wird aber erst nach vollständiger Zahlung Eigentümer. Zinsen sind meist in den Raten enthalten.

\textbf{Darlehen}

Ein langfristiger Kredit wird auch Darlehen genannt. Es gibt einen bestimmten Zinssatz und eine bestimmte Tilgungsregelung (Wie der Darlehensbetrag exkl. Zinsen gezahlt wird).

Es gibt folgende Arten von Darlehen:

\begin{itemize}
    \item Fälligkeitsdarlehen: Während der Laufzeit werden feste Zinsen gezahlt und am Ende erfolgt die Tilgung auf einen Schlag.
    \item Annuitätendarlehen: Es werden feste Raten (gen. Annuitäten) gezahlt. Dafür sinkt der Zinsanteil und Tilgungsanteil steigt.
    \item Abzahlungsdarlehen: Es werden Raten mit gleichbleibenden Tilgungsbeträgen gezahlt. Die Zinsen sinken je Rate mit der Restschuld.
\end{itemize}

\textbf{Leasing}

Leasing ist das eigentliche Überlassen (Vermieten) von Gegenständen des Anlegevermögens über einen bestimmten Zeitraum. Der Leasinggeber (Vermieter) übergibt dem Leasingnehmer (Mieter) den Gegenstand gegen eine Leasingrate (Miete). Nach Ablauf des Leasingvertrages kann der Leasingnehmer den Gegenstand kaufen, zurückgeben oder möglicherweise weiter leasen.

Beim Leasing ist der Kapitalbedarf geringer.

Leasingraten für Anlagegut kann als Aufwand in der Gewinn- und Verlustrechnung verbucht werden. Dies ist auch für die jährliche Abschreibungen (Wertminderungen) eines Kredites und für Fremdkapitalzinsen möglich.

Ein Leasingvertrag kann regeln Technologien und Anlagen gegen neuere Modelle auszutauschen.

Leasing von Anlagegütern beinhaltet meist auch die Dienstleistung der Wartung und Reparatur.

Leasing ist i.d.R. teurer als ein Kauf - selbst auf Kredit.