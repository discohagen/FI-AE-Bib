\section{Nutzwertanalyse}
Die Nutzwertanalyse (NWA) ist eine Methode des qualitativen (Angebots-)Vergleich, indem unterschiedlich bestimmte Teilnutzen zu einem Gesamtnutzen addiert werden, welcher gegen Alternativen verglichen werden kann.

\textbf{Vorgehensweise}

\begin{enumerate}
    \item Festlegen der Bewertungskriterien / Teilnutzenaspekten
    \item Festlegen der Gewichtungsfaktoren / Anteile der einzelnen Bewertungskriterien
    \item Aufstellen einer Punkteskala (z.B. Schulnoten, 0-10, 0-100)
    \item Bewertung der Entscheidungsalternativen anhand der Skala
    \item Ermitteln der gewichteten Punktwerte als Produkt aus Bewertung und Gewichtungsfaktor
    \item Summieren der gewichteten Punktwerte
    \item Interpretation der Ergebnisse
\end{enumerate}

\textbf{Beispiel}

\begin{table}[H]
    \centering
    \begin{tabularx}{\textwidth}{|>{\arraybackslash}X|c|>{\centering\arraybackslash}X|>{\centering\arraybackslash}X|>{\centering\arraybackslash}X|>{\centering\arraybackslash}X|>{\centering\arraybackslash}X|>{\centering\arraybackslash}X|}
        \hline
                    &       & \multicolumn{2}{c|}{Unternehmen 1} & \multicolumn{2}{c|}{Unternehmen 2} & \multicolumn{2}{c|}{Unternehmen 3}                                      \\
        \hline
                    & Gew.  & Punkte                             & Gew. Punkte                        & Punkte                             & Gew. Punkte & Punkte & Gew. Punkte \\
        \hline
        Grafikkarte & 20\%  & 3                                  & 60                                 & 2                                  & 40          & 4      & 80          \\
        \hline
        RAM         & 25\%  & 4                                  & 100                                & 3                                  & 75          & 4      & 100         \\
        \hline
        Monitor     & 40\%  & 2                                  & 80                                 & 1                                  & 40          & 4      & 160         \\
        \hline
        Preis       & 15\%  & 3                                  & 45                                 & 4                                  & 60          & 1      & 15          \\
        \hline
                    & 100\% &                                    & 285                                &                                    & 215         &        & 355         \\
        \hline
    \end{tabularx}
    \caption{Beispiel Nutzwertanalyse}
    \label{tab:nutzwertanalyse}
\end{table}

Die Bewertungskriterien bilden das jeweilige Teilnutzen ab. Die Größe eines Nutzens bemisst sich an denjenigen, welche das Gut nutzen, den Zweck, die Situation, der Zeitpunkt und das Gut selbst. Die Bewertungskriterien sollte untereinander nutzenunabhängig sein.

\textbf{Vorteile}

\begin{itemize}
    \item Flexibilität
    \item Schnelligkeit
    \item direkter Vergleich
    \item Eindeutigkeit
\end{itemize}

\textbf{Nachteile}

\begin{itemize}
    \item Subjektivität
    \item Manipulierbarkeit
    \item Ausschluss von Konsequenzen
    \item Niedrige Aussagekraft bei Alternativen mit sehr ähnlichen quantitativen Gesamtnutzen
\end{itemize}