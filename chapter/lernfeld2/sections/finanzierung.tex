\subsection{Finanzierung}

Finanzierung umfasst alle Maßnahmen der Kapitalbeschaffung. Das Kapital kann dabei in Form von Geld, Gütern oder Wertpapieren zur Verfügung stehen.

Finanzierungsarten können durch die Perspektiven der Herkunft der Mittel und der Rechtsstellung des Kapitalgebers differenziert werden.

\begin{table}
    [H]
    \centering
    \begin{tabularx}{\textwidth}{|l|X|X|X|}
        \cline{3-4}
        \multicolumn{2}{c|}{}            & \multicolumn{2}{c|}{Rechtsstellung des Kapitalgebers}                                                                                                         \\
        \cline{3-4}
        \multicolumn{2}{c|}{}            & Innenfinanzierung                                     & Außenfinanzierung                                                                                     \\
        \hline
        \multirow{3}{*}{Kapitalherkunft} & Eigenfinanzierung                                     & Selbstfinanzierung (aus Gewinn)                               & Einlagen und Beteiligungsfinanzierung \\
        \cline{2-4}
                                         & Fremdfinanzierung                                     & Finanzierung aus Rückstellung                                 & Kreditfinanzierung                    \\
        \cline{2-4}
                                         & Eigen- und Fremdfinanzierung                          & Finanzierung durch Kapitalfreisetzung (Abschreibung, Verkauf) & \multicolumn{1}{c}{}                  \\
        \cline{1-3}
    \end{tabularx}
\end{table}

Selbstfinanzierung ist die Finanzierung aus dem nicht ausgeschütteten Gewinn. Bei Personengesellschaften handelt es sich um Jahresgewinn ohne Privatentnahmen und bei Kapitalgesellschaften um Rücklagen oder Gewinnvortrag.

Einlagen und Beteiligungsfinanzierung wird durch neue Gesellschafter, die Erhöhung der Kapitalanteile (OHG, KG, GmbH), die Erhöhung des Stammkapitals (GmbH) oder die Ausgabe neuer Aktien (AG) erreicht.

Rückstellungen (z.B. für Pensionen) sind Aufwendungen, welch erst in Zukunft zu echten Ausgaben werden. Bis zur Auszahlung kann das zurückgestellte Kapital zu Finanzierungszwecken genutzt werden.

Bei der Außenfinanzierung mit Fremdkapital spricht man von einem Kredit (s.u.).

Finanzierung durch Kapitalfreisetzung ist z.B. durch Abschreibung möglich. Abschreibungen sind Aufwendungen, welche als Bestandteile der erzielten Erlöse wieder in das Unternehmen zurückfließen. Die Abschreibungswerte können bis zum Kauf einer neuen Anlage als zusätzliche Finanzierungsmittel eingesetzt werden.

\textbf{Kredite}

Ein Kredit ist das Überlassen von Geld o.ä. meist gegen Zinsen und der Rückzahlung zu einem bestimmten Zeitpunkt. Der Kreditgeber wird Gläubiger genannt und der Kreditnehmer Schuldner. Außerdem werden i.d.R. Sicherheiten vereinbart, welche der Gläubiger in Anspruch nehmen kann, wenn der Schuldner seinen Pflichten nicht nachkommt.

Es gibt folgende Arten von Krediten.

\textbf{Lieferantenkredit}

Der Lieferer räumt seinem Kunden ein Zahlungsziel zu einem späteren Zeitpunkt ein.

\textbf{Dispositions-/ Kontokorrentkredit}

Kreditinstitute (z.B. Banken) oder auch Lieferer räumen Kunden die Möglichkeit ein, ihr Konto (Girokonto für privat und Kontokorrentkonto für Unternehmen) zu überziehen.

\textbf{Ratenkauf/ Teilzahlung}

Der Rechnungsbetrag wird in Teilen beglichen. Der Käufer erhält die Ware sofort, wird aber erst nach vollständiger Zahlung Eigentümer. Zinsen sind meist in den Raten enthalten.

\textbf{Darlehen}

Ein langfristiger Kredit wird auch Darlehen genannt. Es gibt einen bestimmten Zinssatz und eine bestimmte Tilgungsregelung (Wie der Darlehensbetrag exkl. Zinsen gezahlt wird).

Es gibt folgende Arten von Darlehen:

\begin{itemize}
    \item Fälligkeitsdarlehen: Während der Laufzeit werden feste Zinsen gezahlt und am Ende erfolgt die Tilgung auf einen Schlag.
    \item Annuitätendarlehen: Es werden feste Raten (gen. Annuitäten) gezahlt. Dafür sinkt der Zinsanteil und Tilgungsanteil steigt.
    \item Abzahlungsdarlehen: Es werden Raten mit gleichbleibenden Tilgungsbeträgen gezahlt. Die Zinsen sinken je Rate mit der Restschuld.
\end{itemize}

\textbf{Leasing}

Leasing ist das eigentliche Überlassen (Vermieten) von Gegenständen des Anlegevermögens über einen bestimmten Zeitraum. Der Leasinggeber (Vermieter) übergibt dem Leasingnehmer (Mieter) den Gegenstand gegen eine Leasingrate (Miete). Nach Ablauf des Leasingvertrages kann der Leasingnehmer den Gegenstand kaufen, zurückgeben oder möglicherweise weiter leasen.

Beim Leasing ist der Kapitalbedarf geringer.

Leasingraten für Anlagegut kann als Aufwand in der Gewinn- und Verlustrechnung verbucht werden. Dies ist auch für die jährliche Abschreibungen (Wertminderungen) eines Kredites und für Fremdkapitalzinsen möglich.

Ein Leasingvertrag kann regeln Technologien und Anlagen gegen neuere Modelle auszutauschen.

Leasing von Anlagegütern beinhaltet meist auch die Dienstleistung der Wartung und Reparatur.

Leasing ist i.d.R. teurer als ein Kauf - selbst auf Kredit.