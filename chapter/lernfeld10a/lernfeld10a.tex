\chapter{Lernfeld 10a: Benutzerschnittstellen gestalten und entwickeln}

\textbf{Die Schülerinnen und Schüler verfügen über die Kompetenz, Benutzeroberflächen für
softwarebasierte Arbeitsabläufe und Geschäftsprozesse zu gestalten und zu entwickeln.}

Die Schülerinnen und Schüler \textbf{informieren} sich über die vorhandenen betrieblichen Abläufe
und Geschäftsprozesse.

Sie \textbf{stellen} diese modellhaft \textbf{dar} und leiten Optimierungsmöglichkeiten ab.

Sie \textbf{gestalten} und \textbf{entwickeln} mit agilen Methoden die Benutzeroberflächen für unterschiedliche Endgeräte und Betriebssysteme und stellen die vollständige Abbildung des Informationsflusses unter Berücksichtigung der Prozessbeschreibung sicher.

Die Schülerinnen und Schüler \textbf{stellen} die Funktionalität der Softwarelösung \textbf{her} und nutzen
hierzu bereits vorhandene Bibliotheken und Module.

Sie \textbf{überprüfen} das Produkt auf Datenschutzkonformität und Benutzerfreundlichkeit.

Die Schülerinnen und Schüler \textbf{testen} die funktionale Richtigkeit. Sie quantifizieren die Reduktion der Prozesskosten des digitalisierten, optimierten Geschäftsprozesses und stellen
diese den Entwicklungskosten gegenüber. 