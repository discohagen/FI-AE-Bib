\chapter{Lernfeld 8: Daten systemübergreifend bereitstellen}

\textbf{Die Schülerinnen und Schüler besitzen die Kompetenz, Daten aus dezentralen Quellen zusammenzuführen, aufzubereiten und zur weiteren Nutzung zur Verfügung zu
stellen.}

Die Schülerinnen und Schüler ermitteln für einen Kundenauftrag Datenquellen und \textbf{analysieren} diese hinsichtlich ihrer Struktur, rechtlicher Rahmenbedingungen, Zugriffsmöglichkeiten und -mechanismen.

Sie \textbf{wählen} die Datenquellen (heterogen) für den Kundenauftrag \textbf{aus}.

Sie \textbf{entwickeln} Konzepte zur Bereitstellung der gewählten Datenquellen für die weitere Verarbeitung unter Beachtung der Informationssicherheit.

Die Schülerinnen und Schüler \textbf{implementieren} arbeitsteilig, auch ortsunabhängig, ihr Konzept mit vorhandenen sowie dazu passenden Entwicklungswerkzeugen und Produkten.

Sie \textbf{übergeben} ihr Endprodukt mit Dokumentation zur Handhabung, auch in fremder Sprache, an die Kunden.

Sie \textbf{reflektieren} die Eignung der eingesetzten Entwicklungswerkzeuge hinsichtlich des arbeitsteiligen Entwicklungsprozesses und die Qualität der Dokumentation. 

\section{ERM (Entity Relationship Model)}

\section{SQL}