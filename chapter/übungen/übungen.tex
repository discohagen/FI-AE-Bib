\chapter{Übungen}

\section{Netzwerke}

\subsection{IPv4}

\subsubsection{1.}

Gegeben ist die IPv4-Adresse 10.20.30.0/21.

\textbf{a.} Binäre Darstellung

\begin{itemize}
    \item IP: 00001010.00010100.00011110.00000000
    \item NM: 11111111.11111111.11111000.00000000
\end{itemize}

\textbf{b.} Netz- und Hostteil

\begin{table}
    [H]
    \centering
    \begin{tabular}{lr|l}
        IP: & 00001010.00010100.00011 & 110.00000000 \\
        NW: & 11111111.11111111.11111 & 000.00000000 \\
            & 21 bit                  & 11 bit       \\
            & Netzteil                & Hostteil     \\
    \end{tabular}
    \caption{IPv4 1. Netz- und Hostteil}
\end{table}

\textbf{c.} Netzwerkadresse (NW)

\begin{table}
    [H]
    \centering
    \begin{tabular}{ll}
        IP:                 & 00001010.00010100.00011110.00000000 \\
        NM:                 & 11111111.11111111.11111000.00000000 \\
        IP $\land$ NM = NW: & 00001010.00010100.00011000.00000000 \\
    \end{tabular}
    \caption{IPv4 1. Netzwerkadresse}
\end{table}

Oder dezimal: 10.20.24.0

\textbf{d.} Broadcastadresse (BC)

Die Broadcastadresse ergibt sich aus der Netzwerkadresse mit allen Hostbits auf 1:

BC binär: 00001010.00010100.00011$|$111.11111111

BC dezimal: 10.20.31.255

\textbf{e.} Kleinste und Größte Hostadresse

Die kleinste Hostadresse ist die Netzwerkadresse + 1, die größte Hostadresse ist die Broadcastadresse - 1.

Kleinste Hostadresse (binär): 00001010.00010100.00011$|$000.00000001

Kleinste Hostadresse (dezimal): 10.20.24.1

Größte Hostadresse (binär): 00001010.00010100.00011$|$111.11111110

Größte Hostadresse (dezimal): 10.20.31.254

\textbf{f.} Anzahl der IPv4 Adressen und Hosts

Die Anzahl der IPv4 Adressen ergibt sich aus der Anzahl der Hostbits oder dem Stellenwert des niederwertigsten Bits der Subnetzmaske.

Anzahl IPs: $2^{11} = 2048$

Anzahl Hosts: $2^{11} - 2 = 2046$

\subsubsection{2.}

Gegeben sind die folgenden IP-Adressen:

\begin{itemize}
    \item 12.13.14.16/25
    \item 12.13.14.32/25
\end{itemize}

Liegen die beiden Adressen im selben Subnetz?

\begin{enumerate}
    \item Binäre Darstellung bestimmen: \\ 00001100.00001101.00001110.00010000 und \\ 00001100.00001101.00001110.00100000
    \item Subnetzmaske bestimmen: \\ 11111111.11111111.11111111.1000000
    \item Netzwerkadressen berechnen: \\ 00001100.00001101.00001110.00000000 = 12.13.14.0 und \\ 00001100.00001101.00001110.00100000 = 12.13.14.32
    \item Vergleich der Netzwerkadressen $\Rightarrow$ Die Netze liegen nicht im gleichen Subnetz $\square$
\end{enumerate}